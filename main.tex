\documentclass{llncs}
%
\usepackage{makeidx}  % allows for indexgeneration

\usepackage{amssymb}
\usepackage{natbib}

% Typographie française ==================================================================
\usepackage[english]{babel} % typographie francaise
\usepackage[utf8]{inputenc} % saisie de caracteres accentues, etc.

% Vos environnements =====================================================================
\newtheorem{prop}{Proposition}
\newenvironment{pf}{{\bf Proof }}{\hfill$\Box$\par}
\newcommand{\spaceafterproof}{\vspace{1em}}

% Corps de l'article =====================================================================
\usepackage{color}
% the following package is optional:
%\usepackage{latexsym} 
%\usepackage{amsthm}
%% \newtheorem{theorem}{Theorem}
%% \newtheorem{lemma}{Lemma}
\usepackage{amsmath}
\usepackage{graphicx}
\usepackage{subfigure}
%\usepackage{algorithm2e}
\usepackage{algorithm}
\usepackage{algorithmic}
\newcommand{\argmax}{\operatorname*{argmax}} %\operatorname* pour les op. pouvant admettre des limites...

\title{Apprentissage par imitation dans un cadre \emph{batch}, \emph{off-policy} et sans modèle}
%\author{Robert}
%   1. Supelec-Metz Campus, IMS Reasearch group, France. \texttt{prenom.nom@supelec.fr}}
\author{Edouard Klein\inst{1}\inst{3} \and Matthieu Geist\inst{1} \and Olivier Pietquin\inst{1}\inst{2}}
 \institute{
 1. Supélec-Metz Campus, IMS Research group, France, \texttt{prenom.nom@supelec.fr}\\
 2. UMI 2958 CNRS - GeorgiaTech, France\\
 3. Equipe ABC, LORIA-CNRS, France.
 }
\date{\today}

\begin{document}


\maketitle
\begin{abstract}
  This paper addresses the problem of apprenticeship learning, that
  is learning control policies from demonstration by an expert. An
  efficient framework for it is inverse reinforcement learning
  (IRL). Based on the assumption that the expert maximizes a utility
  function, IRL aims at learning the underlying reward from example 
  trajectories. Many IRL algorithms
  assume that the reward function is linearly parameterized and rely
  on the computation of some associated \emph{feature expectations},
  which is done through Monte Carlo simulation. However, this
  assumes to have full trajectories for the expert policy as well as
  at least a generative model for intermediate policies. In this
  paper, we introduce a temporal difference method, namely LSTD-$\mu$, to compute these
  feature expectations. This allows extending apprenticeship
  learning to a batch and off-policy setting.
\end{abstract}
\section{Introduction}

Optimal control consists in putting a machine in control of a system
with the goal of fulfilling a specific task, optimality being
defined as how well the task is performed. Various solutions to this
problem exist from automation to planification. Notably, the
reinforcement learning (RL) paradigm~\citep{sutton1998reinforcement} is a general
machine learning framework in which an agent learns to control
optimally a dynamic system through interactions with it. The task is
specified through a reward function, the agent objective being to
take sequential decisions so as to maximize the expected cumulative
reward.

However, defining optimality (through the reward function) can
itself be a challenge. If the system can be empirically controlled
by an expert, even though his/her behavior can be difficult to
describe formally, apprenticeship learning is a way to have a
machine controlling the system by mimicking the expert. Rather than
directly mimicking the expert with some supervised learning
approach, Inverse Reinforcement Learning (IRL)~\citep{ng2000algorithms}
consists in learning a reward function under which the policy
demonstrated by the expert is optimal. Mimicking the expert therefore
ends up to learning an optimal policy according to this reward
function. A significant advantage of such an approach is that
expert's actions can be guessed in states which have not been
encountered during demonstration. Firstly introduced in \citep{russell1998learning}, another advantage claimed by the author would be to
find a compact and complete representation of the task in the form
of the reward function.

There roughly exists three families of IRL algorithms:
feature-expectation-based~\citep{abbeel2004apprenticeship,syed2008apprenticeship,syed2008game,ziebart2008maximum},
margin-maximization-based~\citep{ratliff2006maximum,ratliff2007imitation,ratliff2007boosting,kolter2008hierarchical}
and approaches based on the parameterization of the policy by the
reward function~\citep{ramachandran2007bayesian,neu2007apprenticeship}. The first family
assumes a linearly parameterized reward function. This naturally
leads to a linearly parameterized value function, the associated
feature vector being the so-called feature expectation (see Section~\ref{sec:back} for a formal definition).
These approaches learn a reward function such that the feature
expectation of the optimal policy (according to the learnt reward
function) is close to the feature expectation of the expert policy.
This is a sufficient condition to have close value functions (for any parameterized reward function, and therefore particularly the optimized one). The
second family expresses IRL as a constrained optimization problem in
which expert's examples have higher expected cumulative reward than all other
policies by a certain margin. Moreover, suboptimality of the expert
can be considered through the introduction of slack variables. The
last family parameterizes policies with a reward function. Assuming
that the expert acts according to a Gibbs policy (respectively to
the optimal value function related to the reward function which is
optimized), it is possible to estimate the likelihood of a set of
state-action pairs provided by the expert. The algorithms differ in
the way this likelihood is maximized.

This paper focuses on the first family of algorithms, and more
precisely on the seminal work of~\citet{abbeel2004apprenticeship}. All of them
rely on the computation of the feature expectation (which depends on
policies but not on rewards) of (i) the expert and (ii) some
intermediate policies. The expert's feature expectation is computed
using a simple Monte Carlo approach (which requires full
trajectories of the expert). Other feature expectations are either
computed exactly (which requires knowing analytically the dynamics
of the system) or with a Monte Carlo approach (which requires simulating the system). The contribution of this paper is LSTD-$\mu$,
a new temporal-difference-based algorithm for estimating these feature
expectations. It relaxes the preceding assumptions: transitions
of the expert are sufficient (rather than full trajectories) and nor
the model neither a simulator are necessary to compute intermediate
feature expectations. This paper focuses on the algorithm introduced
in~\citep{abbeel2004apprenticeship}, but the proposed
approach can be used in other algorithms based on feature
expectation computation.

The rest of this paper is organized as follows.
Section~\ref{sec:back} provides the necessary background,
notably the definition of feature expectation and its use in the
seminal IRL algorithm of~\citet{abbeel2004apprenticeship}.
Section~\ref{sec:lstdmu} presents LSTD-$\mu$, our main contribution.
Section~\ref{sec:exp} provides some preliminary experiments and
Section~\ref{sec:conclusion} opens perspectives.


\section{Background}
\label{sec:back}
A sequential decision problem is often framed as a Markov Decision
Process (MDP)~\citep{puterman1994markov}, a tuple
$\{S,A,P,R,\gamma\}$ with $S$ being the state space, $A$ the action
space, $P\in\mathcal{P}(S)^{S\times A}$ the set of Markovian
transition probabilities, $R\in\mathbb{R}^S$ the reward function
(assumed to be absolutely bounded by 1) and $\gamma\in[0,1[$ a
discounting factor. A policy $\pi\in A^S$ maps states to actions. The
quality of a policy is quantified by the associated value function
$V^\pi$, which associates to each state the expected and discounted
cumulative reward:
\begin{equation}
  V^\pi(s) = E[\sum_{t=0}^\infty \gamma^t R(s_t)|s_0=s, \pi]
\end{equation}
Dynamic programming aims at finding the optimal policy $\pi^*$, that
is one of the policies associated with the optimal value function,
$V^* = \argmax_\pi V^\pi$, which maximizes the value for each state.
If the model (that is transition probabilities and the reward
function) is unknown, learning the optimal control through
interactions is addressed by RL.

For IRL, the problem is reversed. It is
assumed that an expert acts according to an optimal policy $\pi_E$,
this policy being optimal according to some unknown reward function
$R^*$. The goal is to learn this
reward function from sampled trajectories of the expert. This is a
difficult and ill-posed problem~\citep{ng2000algorithms}. Apprenticeship
learning through IRL, which is the problem at hand in this paper,
has a somewhat weaker objective: it aims at learning a policy
$\tilde{\pi}$ which is (approximately) as good as the expert policy
$\pi_E$ under the unknown reward function $R^*$, for a known initial
state $s_0$ (this condition can be easily weakened by assuming a
distribution over initial states). Now, the approach proposed in~\citep{abbeel2004apprenticeship} is
presented.

We assume that the true reward function belongs to some hypothesis
space $\mathcal{H}_\phi = \{\theta^T \phi(s),
\theta\in\mathbb{R}^p\}$, of which we assume the basis functions to
be bounded by 1: $|\phi_i(s)|\leq 1, \forall s\in S, 1\leq i \leq
p$. Therefore, there exists a parameter vector $\theta^*$ such that:
\begin{equation}
  R^*(s) = (\theta^*)^T \phi(s)
\end{equation}
In order to ensure that rewards are bounded, we impose that
$\|\theta\|_2\leq 1$. For any reward function belonging to
$\mathcal{H}_\phi$ and for any policy $\pi$, the related value
function $V^\pi(s)$ can be expressed as follows:
\begin{align}
  V^\pi(s) &= E[\sum_{t=0}^\infty \gamma^t \theta^T \phi(s_t)|s_0=s, \pi]
  \\
  &= \theta^T  E[\sum_{t=0}^\infty \gamma^t \phi(s_t)|s_0=s, \pi]
\end{align}
Therefore, the value function is also linearly parameterized, with
the same weights and with basis functions being grouped into the
so-called \emph{feature expectation} $\mu^\pi$:
\begin{equation}
  \mu^\pi(s) = E[\sum_{t=0}^\infty \gamma^t \phi(s_t)|s_0=s, \pi]
\end{equation}
Recall that the problem is to find a policy whose performance is
close to that of the expert's for the starting state $s_0$, on the
unknown reward function $R^*$. In order to achieve this goal, it is
proposed in~\citep{abbeel2004apprenticeship} to find a policy $\tilde{\pi}$ such
that $\|\mu^{\pi_E}(s_0)-\mu^{\tilde{\pi}}(s_0)\|_2\leq \epsilon$
for some (small) $\epsilon>0$. Actually, this ensures that the value
of the expert's policy and the value of the estimated policy (for
the starting state $s_0$) are close for \emph{any} reward function
of $\mathcal{H}_\phi$:
\begin{align}
  |V^{\pi_E}(s_0) - V^{\tilde{\pi}}(s_0)| &=
  |\theta^T(\mu^{\pi_E}(s_0)-\mu^{\tilde{\pi}}(s_0))|
  \label{eqn:vs0}
  \\
  &\leq \|\mu^{\pi_E}(s_0)-\mu^{\tilde{\pi}}(s_0)\|_2
\end{align}
This equation uses the Cauchy-Schwarz inequality and the
assumption that $\|\theta\|_2\leq 1$. Therefore, the approach
described here does not ensure to retrieve the true reward function
$R^*$, but to act as well as the expert under this reward function
(and actually under any reward function).

Let us now describe the algorithm proposed in~\citep{abbeel2004apprenticeship} to
achieve this goal:
\begin{enumerate}
  \item Starts with some initial policy $\pi^{(0)}$ and compute
  $\mu^{\pi^{(0)}}(s_0)$. Set $j=1$;
  %
  \item Compute $t^{(j)} = \max_{\theta: \|\theta\|_2\leq 1}
  \min_{k\in\{0,j-1\}}\theta^T(\mu^{\pi_E}(s_0) -
  \mu^{\pi^{(k)}}(s_0))$ and let $\theta^{(j)}$ be the value
  attaining this maximum. At this step, one searches for the reward
  function which maximizes the distance between the value of the
  expert at $s_0$ and the value of \emph{any} policy computed so far
  (still at $s_0$). This optimization problem can be solved using a
  quadratic programming approach or a projection
  algorithm~\citep{abbeel2004apprenticeship};
  %
  \item if $t^{(j)}\leq \epsilon$, terminate. The algorithm outputs a
  set of policies $\{\pi^{(0)}, \dots, \pi^{(j-1)}\}$ among which
  the user chooses manually or automatically the closest to the
  expert (see~\citep{abbeel2004apprenticeship} for details on how to choose this
  policy).
  Notice that the last policy is not necessarily the best (as illustrated in Section~\ref{sec:exp});
  %
  \item solve the MDP with the reward function $R^{(j)}(s) =
  (\theta^{(j)})^T\phi(s)$ and denote $\pi^{(j)}$ the associated
  optimal policy. Compute $\mu^{\pi^{(j)}}(s_0)$;
  %
  \item set $j\leftarrow j+1$ and go back to step 2.
\end{enumerate}
There remain three problems: computing the feature expectation of
the expert, solving the MDP and computing feature expectations of
intermediate policies.

As suggested in~\citep{abbeel2004apprenticeship}, solving the MDP can be done
approximately by using any appropriate reinforcement learning
algorithm. In this paper, we use the Least-Squares Policy Iteration
(LSPI) algorithm~\citep{lagoudakis2003least}. There remains to estimate
feature expectations. In~\citep{abbeel2004apprenticeship}, $\mu^{\pi_E}(s_0)$ is
estimated using a Monte Carlo approach over $m$ trajectories:
\begin{equation}
  \hat{\mu}_E(s_0) = \frac{1}{m} \sum_{h=1}^m \sum_{t=0}^\infty
  \gamma^t \phi(s_t^{(h)})
\end{equation}
This approach does not hold if only transitions of the expert are
available, or if trajectories are too long (in this case, it is
still possible to truncate them). For intermediate policies, it is
also suggested to estimate associated feature expectations using a
Monte Carlo approach (if they cannot be computed exactly). This is
more constraining than for the expert, as this assumes that a
simulator of the system is available. In order to address these
problems, we introduce a temporal-difference-based algorithm to
estimate these feature expectations.

\section{LSTD-$\mu$}
\label{sec:lstdmu}
Let us write the definition of the $i^\text{th}$ component of a
feature expectation $\mu^\pi(s)$ for some policy $\pi$:
\begin{equation}
  \mu_i^\pi(s) = E[\sum_{t=0}^\infty \gamma^t \phi_i(s_t)|s_0=s,\pi]
  \label{eqn:phi}
\end{equation}
This is exactly the definition of the value function of the policy
$\pi$ for the MDP considered with the $i^\text{th}$ basis function
$\phi_i(s)$ as the reward function. There exist many algorithms to
estimate a value function, any of them can be used to estimate
$\mu_i^\pi$. Based on this remark, we propose to use specifically
the least-squares temporal difference (LSTD)
algorithm~\citep{bradtke1996linear} to estimate each component of the
feature expectation (as each of these components can be understood
as a value function related to a specific and known reward
function).

Assume that a set of transitions $\{(s_t,r_t,s_{t+1})_{1\leq t \leq
n}\}$ sampled according to the policy $\pi$ is available. We assume
that value functions are searched for in some  hypothesis space
$\mathcal{H}_\psi = \{ \hat{V}_\xi(s) = \sum\limits_{i=1}^q \xi_i \psi_i(s)
= \xi^T \psi(s), \xi\in\mathbb{R}^q\}$. As reward and value
functions are possibly quite different, another hypothesis space is
considered for value function estimation. But if $\mathcal{H}_\phi$
is rich enough, one can still consider
$\mathcal{H}_\psi=\mathcal{H}_\phi$. Therefore, we are looking for
an approximation of the following form:
\begin{equation}
  \hat{\mu}_i^\pi(s) = (\xi_i^*)^T \psi(s)
  \label{eqn:psi}
\end{equation}
The parameter vector $\xi_i^*$ is here the LSTD estimate:
\begin{equation}
  \xi_i^* = \left(\sum_{t=1}^n
  \psi(s_t)(\psi(s_t)-\gamma\psi(s'_{t}))^T\right)^{-1}
  \sum_{t=1}^n \psi(s_t) \phi_i(s_t)
\end{equation}
For apprenticeship learning, we are interested more particularly in
$\hat{\mu}^\pi(s_0)$. Let $\Psi = (\psi_i(s_t))_{t,i}$ be the
$n\times q$ matrix of values predictors, $\Delta\Psi = (\psi_i(s_t)
- \gamma\psi_i(s'_t))_{t,i}$ be the related $n\times q$ matrix and
$\Phi = (\phi_i(s_t))_{t,i}$ the $n\times p$ matrix of reward
predictors. It can be easily checked that $\hat{\mu}^\pi(s_0)$
satisfies:
\begin{equation}
  (\hat{\mu}^\pi(s_0))^T = \psi(s_0)^T (\Psi^T \Delta\Psi)^{-1} \Psi^T \Phi
\end{equation}
This provides an efficient way to estimate the feature expectation
of the expert in $s_0$.

There remains to compute the feature expectations of intermediate
policies, which should be done in an off-policy manner (that is
without explicitly sampling trajectories according to the policy of
interest). To do so, still interpreting each component of the
feature expectation as a value function, we introduce a state-action
feature expectation defined as follows (much as the classic
$Q$-function extends the value function):
\begin{equation}
  \mu^\pi(s,a) = E[\sum_{t=0}^\infty \gamma^t
  \phi(s_t)|s_0=s,a_0=a,\pi]
\end{equation}
Compared to the classic feature expectation, this definition adds a
degree of freedom on the first action to be chosen before following
the policy $\pi$. With a slightly different definition of the
related hypothesis space, each component of this feature expectation
can still be estimated using the LSTD algorithm (namely using the
LSTD-Q algorithm~\citep{lagoudakis2003least}). The clear advantage of
introducing the state-action feature expectation is that this
additional degree of freedom allows off-policy learning.
Extending LSTD-$\mu$ to state-action LSTD-$\mu$ is done in the same
manner as LSTD is extended to LSTD-Q~\citep{lagoudakis2003least},
technical details are not provided here for the clarity of
exposition.

Given the (state-action) LSTD-$\mu$ algorithm, the apprenticeship
learning algorithm presented in Section~\ref{sec:back} can be easily extended to a batch and
off-policy setting. The solely available data is a set of
transitions sampled according to the expert policy  (and possibly a set of sub-optimal trajectories). The
corresponding feature expectation for the starting state $s_0$ is
estimated with the LSTD-$\mu$ algorithm. At step~4 of this
algorithm, the MDP is (approximately) solved using
LSPI~\citep{lagoudakis2003least} (an approximate policy iteration
algorithm using LSTD-Q as the off-policy $Q$-function estimator).
The corresponding feature expectation at state $s_0$ is estimated
using the proposed state-action LSTD-$\mu$.

Before presenting some experimental results, let us stress that
LSTD-$\mu$ is simply the LSTD algorithm applied to a specific reward
function. Although quite clear, the idea of using a temporal
difference algorithm to estimate the feature expectation is new, as
far as we know. A clear advantage of the proposed approach is that
any theoretical result holding for LSTD also holds for LSTD-$\mu$,
such as convergence~\citep{nedic2003least} or
finite-sample~\citep{lazaric2010finiteLSTD} analysis for example.
\section{Experimental benchmark}
\label{sec:exp}
This section provides experimental results on two complementary problems. Subsection~\ref{ssec:exp} details the protocol and the results while subsection~\ref{ssec:quality} inspects the meaning of the different quality criteria. 
\subsection{Experiment description and results}
\label{ssec:exp}
\subsubsection{GridWorld}
\begin{figure}
\centering
\resizebox{\columnwidth}{!}{\input{../Code/GridWorld/both_error_EB}}
\caption{$||\mu^{\pi}(s_0) - \mu^{\pi_E}(s_0)||_2$ with respect to the number of samples available from the expert. The error bars represent the standard deviation over 100 runs.}
\label{fig:E}
\end{figure}
The first experimental benchmark chosen here is one of those proposed in \citep{ng2000algorithms}, a 5x5 grid world. The agent is in one of the cells of the grid (whose coordinates is the state) and can choose at each step one of the four compass directions (the action). With probability 0.9, the agent moves in the intended direction. With probability 0.1, the direction is randomly drawn. The reward optimized by the expert is 0 everywhere except in the upper-right cell, where it is 1. For every policy, an episode ends when the upper right cell is attained, or after 20 steps. At the start of each episode, the agent is in the lower-left corner of the grid (the opposite of where the reward is). Both the state and action spaces are finite and of small cardinality. Hence, the chosen feature functions $\phi$ and $\psi$ are the typical features of a tabular representation: 0 everywhere except for the component corresponding to the current state (-action pair).\\

Both \citet{abbeel2004apprenticeship}'s algorithm (from now on referred to as the MC variant) and our adaptation (referred to as the LSTD variant) are tested side by side. The MDP solver of the MC variant is LSPI with a sample source that covers the whole grid (each state has a mean visitation count of more than 150) and draws its action randomly. Both $\mu^{\pi^{(j)}}(s_0)$ and $\mu^{\pi_E}(s_0)$ are computed thanks to a Monte-Carlo estimation with enough samples to make the variance negligible. We consider both these computations as perfect theoretical solvers for all intended purpose on this toy problem. We thus are in the case intended by \citet{abbeel2004apprenticeship}. On the other hand the LSTD variant is used without accessing a simulator. It uses LSPI and LSTD-$\mu$, fed only with the expert's transitions (although we could also use non expert transitions  to compute intermediate policies, if available). This corresponds to a real-life setting where data generation is expensive and the system cannot be controlled by an untrained machine.\\

We want to compare the efficiency of both versions of the algorithm with respect to the number of samples available from the expert, as these samples usually are the bottleneck. Indeed as they are quite costly (in both means and human time) to generate they are often not in abundance hence the critical need for an algorithm to be expert-sample efficient. Having a simulator at one's disposal can also be difficult. For the simple problem we use this is however not an issue. The discussion about the choice of the performance metric has its own dedicated subsection (subsection~\ref{ssec:quality}). We use here the $||\mu^{\pi^{(j)}}(s_0) - \mu^{\pi_E}(s_0)||_2$ error term. Fig.~\ref{fig:E} shows, for some numbers of samples from the expert, the value of $||\mu^{\pi^{(j)}}(s_0)-\mu^{\pi_E}(s_0)||_2$ after the algorithm converged or attained the maximum number of iterations (we used 40). The best policy is found by LSTD variant after one iteration only\footnote{Precise reasons why it happens are not clear now, but certainly have something to do with the fact that all the estimations are made along the same distribution of samples.} whereas in the MC variant, convergence happens after at least ten iterations. The best policy is not always the last, and although it has always been observed so far with the LSTD variant, there is absolutely no way to tell whether this is a feature. The score of the best policy (not the last) according to the $||\mu^{\pi^{(j)}}(s_0) - \mu^{\pi_E}(s_0)||_2$ error term is plotted here. We can see that although the LSTD variant is not as good as the MC variant when very few samples are available, both algorithms quickly converge to almost the same value ; our version converges to a slightly lower error value. The fact that our variant can work in a batch, off-policy and model-free setting should make it suitable to a range of tasks inaccessible to the MC variant, the requirement of a simulator being often constraining.
\subsubsection{Inverted Pendulum}
\begin{figure}
\centering
\resizebox{\columnwidth}{!}{\input{../Code/InvertedPendulum/threshold}}
\caption{Number of balancing steps for the policies found by both variants with respect to the number of samples from a sub-optimal policy given to LSPI. Note the absence of middle ground in the quality.}
\label{fig:threshold}
\end{figure}
\begin{figure}
\centering
\resizebox{\columnwidth}{!}{\input{../Code/InvertedPendulum/threshold_EB}}
\caption{Same plot as Fig.~\ref{fig:threshold}, with mean and standard deviation over 100 runs. The big standard deviation stems from the absence of middle ground in the quality of the policies. The increasing mean w.r.t. the abscissa means that the proportion of good policies increases with the number of samples from a sub-optimal policy given to LSPI.}
\label{fig:threshold_EB}
\end{figure}
Another classic toy problem is the \emph{inverted pendulum}, used for example in \citep{lagoudakis2003least} from where we drew the exact configuration (mass of the pendulum, length of the arm, definition of the feature space, etc.). In this setting, the machine must imitate an expert who maintains the unstable equilibrium of an inverted pendulum allowed to move along only one axis. A balancing step is considered successful (reward $0$) when the angle is less than $\pi\over 2$. It is considered a failure (reward $-1$) when the pendulum has fallen (i.e. the angle exceeds $\pi \over 2$). A run is stopped after a failure or after 3000 successful balancing steps.\\

This problem bears two fundamuntal differences with the previous one and thus comes as complementary to it. First, it presents a continuous state space whereas in the GridWorld a tabular representation was possible. Then, the initial state is randomly drawn from a non singleton subspace of the state space. This last point is addressed differently by the LSTD variant and the MC variant. the MC variant will naturally sample the initial state distribution at each new trajectory from the expert. The LSTD variant, on the other hand, still does not need complete trajectories from the expert but mere samples. As it approximates the whole $\mu^{\pi}$ function and not just $\mu^\pi(s_0)$, it is possible to compute the mean of the approximation $\hat\mu^\pi$ over a few states drawn with the same distribution as the initial states.\\

A striking finding on this toy problem is that the parameter controlling the success or the failure of the experiment is not the number of samples from the expert, but the quality of the samples available to LSPI to solve the MDP. \emph{A contrario} to what happened in the previous toy problem, there are fewer samples from the expert than what LSPI needs to solve the MDP as they do not cover the whole state space. When given only these samples, the LSTD variant fails. The MC variant, however, is not bound by such restrictions and can use as many samples as it needs from a simulator. Thus, with only one trajectory from the expert (100 samples) the MC variant is able to successfully balance the pendulum for more than 3000 steps. The problem here stems from the fact that we don't learn the reward, but a policy that maximizes the reward. If the optimal control problem was solved separately, learning the reward only from the samples of the expert would be possible.\\

The sensitivity of both algorithms to the number of samples available to LSPI is extreme, as is shown in Fig.~\ref{fig:threshold}. It may seem nonsensical to restrict the number of samples available to LSPI for the MC variant as it can access a simulator, but this has been done to show that both variants exhibit the same behavior, hence allowing us to locate the source of this behavior in what both algorithms have in common (the use of LSPI inside the structure of \citep{abbeel2004apprenticeship}) excluding the point where they differ (LSTD-$\mu$ and Monte-Carlo).\\

Fig.~\ref{fig:threshold_EB} shows that when given samples from a sub-optimal policy, the LSTD variant can sort the problem out, statistically almost always with 3000 sub-optimal samples, and sometimes with as low as 200 sub-optimal samples. Such a setting is still batch, off-policy and model-free. When given enough sub-optimal samples, both variants are successful with only one trajectory from the expert (i.e. 100 samples, this is what is plotted here). Giving more trajectories from the expert does not improve the success rate.
\subsection{Discussion about the quality criterion}
\label{ssec:quality}
\begin{figure}
\centering
\input{../Code/GridWorld/criteria_mc}
\caption{Different criteria with respect to the number of iterations for a run of the MC variant.}
\label{fig:A}
\end{figure}
\begin{figure}
\centering
\input{../Code/GridWorld/criteria_lstd_EB}
\caption{Different criteria with respect to the number of samples from the expert, for several runs of the LSTD variant. We can see that the algorithm is blind, as all it has access to is always zero. The true error values, however, smoothly converge to something small. Knowing how many expert samples are actually needed in a real world problem is an open question. The error bars represent the standard deviation over 100 runs.}
\label{fig:B}
\end{figure}

Fig.~\ref{fig:A}~and~\ref{fig:B} illustrate (using the GridWorld problem) the difficulty of choosing the quality criterion ; Fig.~\ref{fig:A} shows four different quality criteria during a run of the MC variant. Fig.~\ref{fig:B} shows the same criteria for several runs of the LSTD variant with a varying number of samples from the expert. The $||\mu^{\pi^{(j)}}(s_0) - \mu^{\pi_E}(s_0)||_2$ term is widely discussed in \citep{abbeel2004apprenticeship}'s additional material. It bears an interesting relation with the difference between the expert's value function and the current value function in the initial state with respect \emph{to the current reward} (Eq.~\ref{eqn:vs0}).\\

The fact that  the curves of the true error term $||\mu^{\pi^{(j)}}(s_0) - \mu^{\pi_E}(s_0)||_2$ and  its estimate $||\hat\mu^{\pi^{(j)}}(s_0) - \hat\mu^{\pi_E}(s_0)||_2$ are indistinguishable in Fig.~\ref{fig:A} means that, for it has access to a cheap simulator, the MC variant works as if it had access to the exact values. This however cannot be said of the LSTD variant, for which the two curves are indeed different (Fig.~\ref{fig:B}). Not knowing the true value of $\mu^{\pi_E}(s_0)$ may be a problem for our variant, as it can introduce an error in the stopping criterion of the algorithm.\\

It shall be noted that although it plays its role, the halt criterion is not a good measure of the quality of the current policy in the MC variant either, as it can be low (and thus halt the algorithm) when the policy is bad. The best policy, however, can be easily chosen among all those created during the execution of the algorithm thanks to the $||\mu^{\pi^{(j)}}(s_0) - \mu^{\pi_E}(s_0)||_2$ term, which the MC variant can compute. When this term is low, the objective performance (that is, $V^{\pi_E}(s_0)-V^{\pi^{(j)}}(s_0)$ with respect to the unknown true reward function) is low too. 
\section{Conclusion}
\label{sec:conclusion}
Given some transitions generated by an expert controlling a system and maximizing in the long run an unknown reward, we ported \citet{abbeel2004apprenticeship}'s approach to apprenticeship learning via inverse reinforcement learning to a batch, model-free, off-policy setting. Experimentally, there is a need for either a slightly bigger number of samples from the expert or some samples from a sub-optimal policy. We believe this cost is not prohibitive as our approach only requires isolated samples which are often less difficult to get than whole trajectories as needed by the original approach. Furthermore, tranferring the reward and not the policy may overcome this difficulty. We intend to do this in a real life setting.
The simple idea of using LSTD to estimate the feature expectation could be applied to other algorithms as well, for example \citep{abbeel2004apprenticeship,syed2008apprenticeship,syed2008game,ziebart2008maximum}.\\

%\bibliographystyle{named}
\bibliographystyle{named}
\bibliography{../Biblio/Biblio}
\end{document} 

