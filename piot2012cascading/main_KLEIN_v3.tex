\documentclass{article} % For LaTeX2e
\usepackage{nips12submit_e,times}
\usepackage{makeidx}  % allows for indexgeneration
% For figures
\usepackage{graphicx} % more modern
%\usepackage[latin1]{inputenc}
%\usepackage[francais]{babel}
\usepackage{subfigure}
\usepackage{tabularx}
\usepackage{mathtools}
\usepackage{amsmath}
\usepackage{amssymb}
\usepackage{amsthm}
\newtheorem{definition}{Definition}
\newtheorem{theorem}{Theorem}
\newtheorem{lemma}{Lemma}
\newtheorem{remark}{Remark}
\usepackage{dsfont}
\usepackage{algorithm}
\usepackage{algorithmic}
\usepackage{hyperref}
\hypersetup{
    colorlinks,%
    citecolor=black,%
    filecolor=black,%
    linkcolor=black,%
    urlcolor=black
}
\newcommand{\theHalgorithm}{\arabic{algorithm}}
\mathtoolsset{showonlyrefs=true}
\newtheorem{hypo}{Hypothesis}
\newcommand{\argmax}{\operatorname*{argmax}}
\newcommand{\argmin}{\operatorname*{argmin}}
\newcommand{\arginf}{\operatorname*{arginf}}
\newcommand{\minp}{\operatorname*{min_+}}
\newcommand{\Ker}{\operatorname*{Ker}}
\newcommand{\trace}{\operatorname*{trace}}
\newcommand{\cov}{\operatorname{cov}}
\newcommand{\card}{\operatorname*{Card}}
\newcommand{\vect}{\operatorname*{Vect}}
\newcommand{\var}{\operatorname{Var}}
\newcommand{\diag}{\operatorname{diag}}
\newcommand{\erf}{\operatorname{erf}}
\newcommand{\bound}{\operatorname*{bound}}
\newcommand{\vpi}{\operatorname{VPI}}
\newcommand{\gn}{\operatorname{Gain}}
\newcommand{\p}{\operatorname{Pr}}
\newcommand{\mlp}{\operatorname{MLP}}
\newcommand*\tto[2]{\smash{\mathop{\longrightarrow}\limits_{#1}^{#2}}}
\newcommand*\ntto[2]{\smash{\mathop{\nrightarrow}\limits_{#1}^{#2}}}
\newcommand{\X}{\mathbf{X}}
\newcommand{\Q}{\mathbf{Q}}
\newcommand{\A}{\mathbf{A}}
\newcommand{\Z}{\mathbf{Z}}
\newcommand{\Y}{\mathbf{Y}}
\newcommand{\E}{\mathbf{E}}
\newcommand{\K}{\mathbf{K}}
\newcommand{\F}{\mathcal{F}}
\newcommand{\R}{\mathbf{R}}
\newcommand{\ba}{\mathbf{a}}
\newcommand{\bb}{\mathbf{b}}
\newcommand{\bc}{\mathbf{c}}
\newcommand{\bd}{\mathbf{d}}
\newcommand{\be}{\mathbf{e}}
\newcommand{\af}{\mathbf{f}}
\newcommand{\bg}{\mathbf{g}}
\newcommand{\bh}{\mathbf{h}}
\newcommand{\bi}{\mathbf{i}}
\newcommand{\bj}{\mathbf{j}}
\newcommand{\bk}{\mathbf{k}}
\newcommand{\bl}{\mathbf{l}}
\newcommand{\bm}{\mathbf{m}}
\newcommand{\bn}{\mathbf{n}}
\newcommand{\bo}{\mathbf{o}}
\newcommand{\bp}{\mathbf{p}}
\newcommand{\bq}{\mathbf{q}}
\newcommand{\br}{\mathbf{r}}
\newcommand{\bs}{\mathbf{s}}
\newcommand{\bt}{\mathbf{t}}
\newcommand{\bu}{\mathbf{u}}
\newcommand{\bv}{\mathbf{v}}
\newcommand{\bw}{\mathbf{w}}
\newcommand{\bx}{\mathbf{x}}
\newcommand{\by}{\mathbf{y}}
\newcommand{\bz}{\mathbf{z}}
\newcommand{\ma}{\mathbf{A}}
\newcommand{\mb}{\mathbf{B}}
\newcommand{\mc}{\mathbf{C}}
\newcommand{\md}{\mathbf{D}}
\newcommand{\me}{\mathbf{E}}
\newcommand{\mf}{\mathbf{F}}
\newcommand{\mg}{\mathbf{G}}
\newcommand{\mh}{\mathbf{H}}
\newcommand{\mi}{\mathbf{I}}
\newcommand{\mj}{\mathbf{J}}
\newcommand{\mk}{\mathbf{K}}
\newcommand{\ml}{\mathbf{L}}
\newcommand{\mm}{\mathbf{M}}
\newcommand{\mn}{\mathbf{N}}
\newcommand{\mo}{\mathbf{O}}
\newcommand{\Mp}{\mathbf{P}}
\newcommand{\mq}{\mathbf{Q}}
\newcommand{\mr}{\mathbf{R}}
\newcommand{\ms}{\mathbf{S}}
\newcommand{\mt}{\mathbf{T}}
\newcommand{\Mu}{\mathbf{U}}
\newcommand{\mv}{\mathbf{V}}
\newcommand{\mw}{\mathbf{W}}
\newcommand{\mx}{\mathbf{X}}
\newcommand{\my}{\mathbf{Y}}
\newcommand{\mz}{\mathbf{Z}}
\newcommand{\tphi}{\tilde{\Phi}}
\newcommand{\espace}{\text{ }}
\newcommand{\x}{\mathbf{x}}
\newcommand{\s}{\mathbf{s}}
\newcommand{\n}{\mathbf{n}}
\newcommand{\y}{\mathbf{y}}
\newcommand{\I}{\mathbf{I}}
\newcommand{\rr}{\mathbf{r}}
\newcommand{\0}{\mathbf{0}}
\newcommand{\1}{\mathbf{1}}
\newcommand{\am}{{\mathcal{A}_m}}
\newcommand{\amj}{{\mathcal{A}_m^{+j}}}
\newcommand{\sgn}{\operatorname{sgn}}
\title{Learning a reward function from demonstrations: a cascaded supervised
learning approach}
\author{Edouard Klein$^{1,2}$\\
 1. ABC Team\\
 LORIA-CNRS, France.
\And Bilal Piot$^{2}$\\
 2. Supélec-Metz Campus\\
 MaLIS Research group, France\\
\And Matthieu Geist$^1$\\
\texttt{prenom.nom@supelec.fr}\\
\And Olivier Pietquin$^{2,3}$\\
3. UMI 2958 CNRS\\
GeorgiaTech, France
}

% The \author macro works with any number of authors. There are two commands
% used to separate the names and addresses of multiple authors: \And and \AND.
%
% Using \And between authors leaves it to \LaTeX{} to determine where to break
% the lines. Using \AND forces a linebreak at that point. So, if \LaTeX{}
% puts 3 of 4 authors names on the first line, and the last on the second
% line, try using \AND instead of \And before the third author name.

\newcommand{\fix}{\marginpar{FIX}}
\newcommand{\new}{\marginpar{NEW}}

\begin{document}

\maketitle              % typeset the title of the contribution
\begin{abstract}
This paper considers the inverse reinforcement learning (IRL) problem. The IRL framework assumes that an expert, demonstrating a task, is acting optimally in a Markov Decision Process (MDP) with respect to an unknown reward function to be discovered. This reward function is seen as the most succinct description of the task, allowing for task transfer from the expert to an agent with potentially different abilities. The proposed contribution consists in cascading a classification and a regression steps to produce a non-trivial reward function for which we show the expert policy to be near optimal. In addition to being generic, this approach is model-free, it does not require solving any direct reinforcement learning problem (unlike most of IRL algorithms) and it solely relies on transitions from the expert (no need to sample trajectories according to other policies). All of this is illustrated through nontrivial experiments.
\end{abstract}

\section{Introduction}

Given a sequential decision making problem posed in the Markov Decision Process (MDP) formalism, it is often difficult to specify manually a reward function in order to realize a given task. However, observing an expert demonstrating a task is often possible and easier~\cite{ng2000algorithms}. That is why learning from demonstrations and more particularly Inverse Reinforcement Learning (IRL) has received more and more interest this last decade. The IRL framework, first introduced in~\cite{russell1998learning},\cite{ng2000algorithms}, assumes that an expert, demonstrating a task, is acting optimally in a MDP with respect to an unknown reward function to be discovered. Unlike direct methods of learning from demonstrations which learn a mapping from states to actions via supervised learning(SL)~\cite{atkeson1997robot},\cite{pomerleau1989alvinn}, IRL methods aim at recovering a reward function which is the most compact and transferable way to specify a task. Moreover most of the direct methods do not use the structure of the MDP (one notable exception is~\cite{melo2010learning}), hence mimic the policy of the expert and are not able to generalize correctly the expert policy outside state examples. In order to overcome this drawback, undirect methods known as apprenticeship learning~\cite{abbeel2004apprenticeship} were introduced. These methods (see~\cite{neu2009training} for a comprehensive overview) uses the IRL framework but most of them assume the dynamic of the environment known, or an environment model (via a simulator) must be available so as to test the effects of a policy. Moreover, most of existing undirect approaches require to solve the direct reinforcement learning problem (find an optimal policy knowing a given reward) several times (one notable exception is~\cite{boularias2011relative}). Finally, in many cases, the output of the algorithm is a policy but not a reward function (see Section \ref{section: related work} for an in-depth presentation). These constraints can lead to the development of algorithms unapplicable to real-world applications and make the problem of apprenticeship learning harder than reinforcement learning. That is why a reborn interest for direct methods which passively imitate the expert exists. More particularly the reduction of the apprenticeship learning to classification, first introduced in~\cite{zadrozny2003cost} and recently used in~\cite{melo2010learning}, is legitimated for finite horizons problems in~\cite{syed2010reduction},\cite{ross2010efficient}. Our approach tries to avoid the drawbacks of the direct and undirect methods by cascading two well known supervised learning(see Section \ref{section: Cascading}): first a step of classification and then a step of regression which introduces the MDP structure and  outputs a reward function (see Section \ref{section: Cascading}). Our approach doesn't require any of the following: complete trajectories from an expert, a generative model of the environment, the knowledge of the transition probabilities, the ability to compute a (near)-optimal policy for different rewards and hence resolve the forward reinforcement learning (RL) problem, the perfect knowledge of the expert's policy. In addition, only a small amount of expert demonstrations (not even in the form of trajectories but simple transitions) is required.\\
The remainder of the paper is organized as follows: Section \ref{section: background} gives us the different notations used in the paper and presents briefly the RL and IRL frameworks, Section \ref{section: Cascading} present our method which is legitimated in Section \ref{section: Cascading}, Section \ref{section: experiments} presents an instantiation of our method and tests it on different benchmarks and finally Section \ref{section: related work} compares our method to the state of the art.

\section{Background and Notations}
\label{section: background}
A (finite) MDP \cite{puterman1994markov} is a tuple $M=\{S,A,P,\gamma,R\}$ where $S=\{s_i\}_{1\leq i \leq N}$ is the finite state space with $N\in\mathbb{N}^*$ states, $A=\{a_k\}_{1\leq k \leq K}$ is the finite action space with $K\in\mathbb{N}^*$ actions, $P=\{P_{s,a}\}_{(s,a)\in S\times A}$ is the set of transition probabilities where $P_{s,a}$ is a distribution probability over $S$, $P$ represents the dynamics of the MDP and the notation $p(s,a,s')=P_{s,a}(s')$ is often used and quantifies the probability to reach $s'\in S$ knowing that the action $a \in A$ was taken in the state $s\in S$, $\gamma\in]0,1[$ is the discount factor and $R$ is a function from $S\times A$ to $\mathbb{R}$ called the reward function. A deterministic policy $\pi$ is a function from $S$ to $A$, $\Pi$ is the set of deterministic policies and the value function $V^\pi_R$ is a function from $S$ to $\mathbb{R}$ defined by:
\begin{equation}
V^\pi_R(s)=E^\pi_s[\sum_{t=0}^{+\infty}\gamma^tR(s_t,\pi(s_t))], \forall s \in S,
\end{equation}
where $E^\pi_s$ is the expectation over the distribution of the trajectories $(s_0,s_1,\dots)$ obtained by executing the policy $\pi$ starting from $s_0=s$.
The action-value function $Q^\pi_R$ is a function from $S\times A$ to $R$ defined by:
\begin{equation}
Q^\pi_R(s,a)=E^\pi_{s,a}[R(s_0,a)+\sum_{t=1}^{+\infty}\gamma^tR(s_t,\pi(s_t))], \forall s \in S,\forall a \in A,
\end{equation}
where $E^\pi_{s,a}$ is the expectation over the distribution of the trajectories $(s_0,s_1,\dots)$ obtained starting from $s_0=s$, executing $a$ and then following the policy $\pi$.
The classical results for finite MDPs are the well-known Bellman equations \cite{sutton1998reinforcement}:
\begin{align}
\label{equation: Bellman}
&V^{\pi}_R(s)=R(s,\pi(s))+\gamma\sum_{s'\in S}P_{s,\pi(s)}(s')V^{\pi}_R(s'), \forall s\in S,
\\
&Q^{\pi}_R(s,a)=R(s,a)+\gamma\sum_{s'\in S}P_{s,a}(s')V^{\pi}_R(s'), \forall s\in S, \forall a\in A.
\end{align}
A policy $\pi$ is said optimal for the reward $R$ when:
\begin{equation}
\label{equation:Voptimal}
V^{\pi}_R(s)\geq V^{\tilde{\pi}}_R(s) , \forall s\in S, \forall \tilde{\pi}\in\Pi.
\end{equation}
Another classical result called the Bellman optimality is: $\pi$ is an optimal policy for the reward $R$ if and only if:
\begin{equation}
\label{equation:Qoptimal}
\pi(s)\in\argmax_{a\in A} Q^\pi_R(s,a), \forall s\in S.
\end{equation}
For a finite-MDP $M=\{S,A,P,\gamma,R\}$, the existence of an optimal policy for any function $R$ is shown in \cite{bertsekas2001dynamic}.
As $S=\{s_i\}_{1\leq i \leq N}$ and $A=\{a_i\}_{1\leq i \leq K}$ are finite, a function $X$ from $S$ to $\mathbb{R}$ can be identified to the real column-vector $[X(s_i)]_{1\leq i \leq N}$ and a function $Y$ from
$S\times A$ to $\mathbb{R}$ can be identified to the real rectangular-matrice of $N$ rows and $K$ columns  $[Y(s_i,a_k)]_{1\leq i \leq N, 1\leq k \leq K}$.
Thus the notation $V^\pi_R$ depending on the context will be considered as a function or a column vector. Moreover if $Y$ is a function from $S\times A$ to $\mathbb{R}$ then the notation $Y_a$ will mean that $Y_a$ is the function from $S$ to $R$ that maps $s$ to $Y(s,a)$. So the notation $R_a$ can be a function or a column-vector depending on the context.
We will also need the following matrices and column-vectors defined for any dynamic $P$, any policy $\pi$ and reward function $R$ :$P_a=[P_{s_i,a}(s_j)]_{1\leq i,j \leq N}$, $P_\pi=[P_{s_i,\pi(s_i)}(s_j)]_{1\leq i,j \leq N}$, $R_\pi=[R(s_i,\pi(s_i))]_{1\leq i,j \leq N}$. The transposition of a real column-vector $X=[X_i]_{1\leq i \leq I}$, $I\in\mathbb{N}^*$, is the real row-vector $X^T$ and $\|X\|_{\infty}=\max_{1\leq i \leq I}|X_i|$. The transposition of the real matrice $Y=[Y_{i,j}]_{1\leq i \leq I, 1\leq j \leq J}$, $(I,J)\in\mathbb{N}^2$, will be $Y^T$ and $\|Y\|_{\text{max}}=\max_{1\leq i \leq I,1\leq j \leq J}|Y_{i,j}|$.  Thanks to these vectorial notations, we can define the Bellman operator $T^\pi_R$, for a given deterministic policy $\pi$ and a given reward $R$, which is a function from $\mathbb{R}^N$ to $\mathbb{R}^N$ such that:
\begin{equation}
T^\pi_RX=R_\pi+\gamma P_\pi X , \forall X\in \mathbb{R}^N.
\end{equation}
It is well known that the unique fixed point of the operator $T^\pi_R$ is the column-vector $V^\pi_R$ \cite{puterman1994markov}.\\
RL consists in finding an optimal policy for the reward function $R$. Notice that the state space may be too large for an exact representation of the value function (which calls for approximate
representation), that the model ($P$ and $R$) may be unknown (the only information being provided through rewarded transitions sampled according
to some behavioral policy), that learning can occurs in an online or off-line setting, and so on. There exists books on the subject \cite{bertsekas2001dynamic},\cite{sutton1998reinforcement}.\\
In the classical IRL paradigm \cite{ng2000algorithms}, an MDP without reward $M\backslash R =\{S,A,P,\gamma\}$ and a policy $\pi_E$ called expert-policy are given and the problem is to find
a reward $R^*$ for which the policy $\pi_E$ is optimal. However this problem is clearly ill-posed in the sense that there is not uniqueness of the reward $R^*$ : many rewards functions are equivalent in the sense that they have the same optimal deterministic policies \cite{ng1999policyreward}, moreover the trivial zero-reward is a solution for any deterministic policy $\pi_E$ as it is shown in~\cite{ng2000algorithms}. In the literature, some solutions are proposed in order to respond to the ill-posed nature of the problem \cite{ng2000algorithms},\cite{ziebart2008maximum},\cite{boularias2011relative}.
In our experiments, see Section \ref{section: experiments}, we assume that the solely available information is provided by transitions sampled according to the dynamics of the environnement under $\pi_E$:
\begin{equation}
\{(s_i,a_i=\pi_E(s_i)),s_i')\}_{1\leq i \leq D}, D\in\mathbb{N}^*,
\end{equation}
where $s_i'$ is sampled according to the distribution $P_{s_i,a_i}$.
The reward function is obviously unknown, but this assumption means that the dynamics ($P$) is only known through transitions $(s_i, a_i, s_i')$ and that the
policy $\pi_E$ is only known through state-action pairs $(s_i, a_i)$.\\
For a given deterministic policy $\pi$, a component $P_{s_i,\pi(s_i)}(s_j)$ of the matrix $P_\pi$ represents the probability to transit from $s_i$ to $s_j$ under the policy $\pi$. So $P_\pi$ can be seen as a transition matrix of a finite Markov-chain on the finite state space $S$. Thus, let us recall some basic but important results of finite-Markov chain theory.
\begin{definition}
Let $Y=[Y_{i,j}]_{1\leq i,j \leq M}$ be a  stochastic square matrix of size $M\in\mathbb{N}^*$. Then $Y$ is said to be irreducible if:
\begin{equation}
U(i,j)=\sum_{k=0}^{+\infty}Y^k(i,j)>0 ,\forall (i,j)\in \{1,\dots,M\}^2.
\end{equation}
\end{definition}
\begin{theorem}[\cite{baldi2000martingales}]
Let $Y$  be a stochastic square matrix of size $M\in\mathbb{N}^*$. $Y$ is irreducible if and only if there exists a unique and strictly positive distribution $\mu$ over $\{1,\dots,M\}$, $\mu$ is a function from $\{1,\dots,M\}$ to $\mathbb{R}$ which can be seen as a real column vector of size $M$, such that:
\begin{equation}
\label{equation: stationarity}
\mu^T=\mu^TY.
\end{equation}
$\mu$ is called the stationary distribution of $Y$.
\end{theorem}
\begin{definition}
Let $Y$ be a stochastic square matrix of size $M\in\mathbb{N}^*$ and irreducible. It is aperiodic if for all $(i,j)\in {1,\dots,M}^2$ it exists $n_0\in \mathbb{N}^*$ such that for all $n\geq n_0$: $Y^n(i,j)>0$.
\end{definition}
\begin{theorem}[\cite{baldi2000martingales}]
\label{theoreme : mixing exponential}
Let $Y$ be a stochastic square matrix of size $M\in\mathbb{N}^*$. If $Y$ is irreducible and aperiodic then:
\begin{equation}
\lim_{k\rightarrow +\infty}\sum_{j=1}^{j=M}[Y^k(i,j)-\mu(j)]\rightarrow 0,\forall i\in\{1,\dots,M\},
\end{equation}
and more precisely there exists $\alpha\in]0,1[$ and $C\in\mathbb{R}_+$ such that:
\begin{equation}
\sum_{j=1}^{j=M}|Y^k(i,j)-\mu(j)|\leq C\alpha^k,\forall i\in\{1,\dots,M\},\forall k\in\mathbb{N}^*.
\end{equation}
\end{theorem}
In our framework, if $P_\pi$ is irreducible, then $\mu_\pi$ will be its stationary distribution.

%Other notations will be needed in order to prove that the cascading approach is legitimate. Let's define the column-vector $\mu^{t,\pi,s}=[\mu^{t,\pi,s}_i]_{1\leq i \leq N}$ where $\mu^{t,\pi,s}_i$ is the probability of being in the state $s_i\in S$ at time $t\in\mathbb{N}$ when the start-state is $s$ and the policy followed is $\pi$ :
%\begin{equation}
%\mu^{t,\pi,s}=(\Delta_s^T(P_\pi)^t)^T,
%\end{equation}
%where $\Delta_s$ is the column-vector: $\Delta_s={\delta(s_i,s)}_{1\leq i \leq N}$ with $\delta(.,.)$ the Kronecker symbol. For a fixed policy $\pi$, the matrix $P_\pi$ is a stochastic matrix of finite dimension and if it is ergodic and recurrent positive then there exists a unique distribution probability represented by the column vector $\mu^\pi=[\mu^{\pi}_i]_{1\leq i \leq N}$ such that :
%\begin{equation}
%\mu_\pi=(\mu_\pi^T P_\pi)^T.
%\end{equation}
%In that case $\mu^\pi$ is called the stationary distribution of the policy $\pi$ and :
%\begin{equation}
%\forall s\in S \lim_{t\rightarrow+\infty}\mu_{t,\pi,s}=\mu_\pi.
%\end{equation}
%Thus in the particular case of ergodicty and recurrent positivity we can easily express $\mu_\pi^TV^\pi_R$ :
%\begin{align}
%\mu_\pi^TV^\pi_R&=\mu_\pi^T(R_\pi+\gamma P_\pi V^\pi_R)=\mu_\pi^TR_\pi+ \gamma\mu_\pi^TP_\pi V^\pi_R,
%\\
%&=\mu_\pi^TR_\pi+ \gamma\mu_\pi^TV^\pi_R=\frac{1}{1-\gamma}\mu_\pi^TR_\pi.
%\end{align}
\section{Cascading Classification and Regression for IRL}
\label{section: Cascading}
In this section, we propose to cascade two supervised learning (SL) approaches (namely a classifier and a regressor) to solve the IRL problem.
\subsection{The general idea}
The available data, provided by the expert, is being depicted by:
\begin{equation}
\label{equation:data}
D_C=\{(s_{C,i},a_{C,i}=\pi_E(s_{C,i})),s'_{C,i})\}_{1\leq i \leq D}, D\in\mathbb{N}^*,
\end{equation}
where $s'_{C,i}$ is sampled according to the distribution $P_{s_{C,i},a_{C,i}}$.
A first problem would be to generalize the expert policy $\pi_E$ outside state examples provided in this dataset. A first approach, recently used in \cite{melo2010learning} and legitimated for finite-horizon problems in \cite{syed2010reduction}, is to frame this problem as learning a multi-classifier where each action $a \in A$ will be a class. Indeed, trained on the training set $\{s_{C,i},a_{C,i}=\pi_E(s_{C,i})\}_{1\leq i \leq D}$, many classifiers are able to output a score function $q$ from $S\times A$ to $\mathbb{R}$. The final decision rule $\pi_C$ which is a function from $S$ to $\mathbb{R}$ and thus can be seen as a deterministic policy is constructed in order to satisfy:
\begin{equation}
\label{equation:Coptimal}
\pi_C(s)\in\argmax_{a\in A}q(s,a),\forall s\in S.
\end{equation}
Now, one can notice the similarity between equation~\eqref{equation:Qoptimal} and equation~\eqref{equation:Coptimal}. The score-function $q$ can then be interpreted as an action-value function and it is easy and natural to construct a reward function $R^C$ for which the policy $\pi_C$ is optimal as it will be proven in the section~\ref{section: Analysis}. Let define $R^C$ as:
\begin{equation}
R^C(s,a)=q(s,a)-\gamma\sum_{s'\in S}p(s,a,s')q(s',\pi_C(s')), \forall a \in A, \forall s\in S.
\end{equation}
However, the model being unknown (contrary to $q$ and $\pi_C$) it is not possible to compute $R^C$ directly. But, it can be estimated from transitions of a regression data set:
\begin{equation}
D_R=\{(s_{R,i},a_{R,i},s'_{R,i})\}_{1\leq i \leq D'}, D'\in\mathbb{N}^*,
\end{equation}
where $s'_{R,i}$ is sampled under the probability $P_{s_{R,i},a_{R,i}}$.
Let us write:
\begin{equation}
\label{ri.def}
\hat{r}_i=q(s_{R,i},a_{R,i})-\gamma q(s'_{R,i},\pi_C(s'_{R,i})), \forall i\in \{1,\dots,D'\}.
\end{equation}
This is an unbiased estimate of $R^C(s_{R,i},a_{R,i})$, indeed:
\begin{equation}
R^C(s_{R,i},a_{R,i})=\sum_{s'\in S}p(s_{R,i},a_{R,i},s')[q(s_{R,i},a_{R,i})-\gamma q(s',\pi_C(s'))], \forall i\in \{1,\dots,D'\}.
\end{equation}
Therefore, it is possible to build an estimate $\hat{R}^C$, which is a function from $S\times A$ to $\mathbb{R}$, of $R^C$ using a regressor trained on the dataset:
\begin{equation}
\{(s_{R,i},a_{R,i},\hat{r}_i)\}_{1\leq i \leq D'}.
\end{equation}
Finally, we see $\hat{R}^C$ as an approximation of $R^C$ and $\hat{\pi}_C$ is defined as an optimal policy for the reward $\hat{R}^C$. In order to verify that the reward function $\hat{R}^C$ is a good candidate to resolve the IRL problem, it will be proven in Section \ref{section: Analysis} that the policy $\pi_E$ is near-optimal for the reward $\hat{R}^C$ (confer theorem \ref{theorem : results}).
\subsection{A particular case}
An interesting case is  when $D_C=D_R$ which is the one chosen in Section \ref{section: experiments}.
In this case, the solely available data are the expert data depicted by the equation \ref{equation:data}.
The approximation function $\hat{R}^C$ obtained thanks to the regression is only accurate on the set $(s,\pi_E(s))_{\{s\in S\}}$ because of the
regression training set.
\section{Analysis} \label{section: Analysis}
This section is devoted to show, under some hypotheses, that the cascading approach is legitimate. The first result is a lemma which gives a practical way to calculate $\mu_\pi^TV^{\pi}_R$ for a given policy $\pi$ and a given reward function $R$. The second result is a theorem which gives an upper bound to the term $\mu_E^TV^{\hat{\pi}_C}_{\hat{R}^C}-\mu_E^TV^{\pi_E}_{\hat{R}^C}$ where $\mu_E$ is the stationary distribution of the expert policy $\pi_E$. We also give an interpretation of the term $\mu_E^TV^{\hat{\pi}_C}_{\hat{R}^C}-\mu_E^TV^{\pi_E}_{\hat{R}^C}$ and explain why being able to bound this term means that our approach is legitimate.
\subsection{Results and Discussion}
\begin{lemma}
\label{lemma: calculs V}
Let $\{S,A,P,\gamma,R\}$ be a finite MDP and $\pi$ a deterministic policy.
If $P_\pi$ is reducible, then $\mu_\pi^TV^\pi_R=\frac{1}{1-\gamma}\mu_\pi^TR_\pi$.
\end{lemma}
The first lemma gives a practical tool which will be useful in order to simplify some terms in the proof of the next theorem. Moreover it is useful
to notice that the term  $\mu_\pi^TV^\pi_R$ can be reinterpreted as an expectation. Indeed we have:
\begin{equation}
\mu_\pi^TV^\pi_R=E_{s \sim \mu_\pi}[V^\pi_R(s)],
\end{equation}
where $E_{s \sim \mu_\pi}$ means that $s$ is distributed over the distribution $\mu_\pi$. All the results provided in the next theorem use the expectation $E_{s \sim \mu_E}$ which is the canonical expectation to consider when one wants to prove a result over the state space $S$ related to the policy of the expert $\pi_E$.\\
Before giving an upper bound to $\mu_E^T(V^{\hat{\pi}_C}_{\hat{R}^C}-V^{\pi_E}_{\hat{R}^C})=E_{s \sim \mu_E}[V^{\hat{\pi}_C}_{\hat{R}^C}(s)-V^{\pi_E}_{\hat{R}^C}(s)]$, we define $\epsilon_C\in\mathbb{R}_+$ called the classification error and the function $\epsilon_R$ from $S\times A$ to $\mathbb{R}$ called the regression error such that:
\begin{align}
&\epsilon_C=\sum_{s\in S}\mu_{E}(s)\mathds{1}_{\{\pi_C(s)\neq\pi_E(s)\}}=E_{s \sim \mu_E}[\mathds{1}_{\{\pi_C(s)\neq\pi_E(s)\}}],
\\
&\epsilon_R(s,a)=\hat{R}^C(s,a)-R^C(s,a), \forall a\in A, \forall s\in S.
\end{align}
The column vectors $\epsilon^C_R=[\epsilon_R(s_i,\hat{\pi}_C(s_i))]_{1\leq i \leq N}$ and $\epsilon^E_R=[\epsilon_R(s_i,\pi_E(s_i))]_{1\leq i \leq N}$ will also be needed.
\begin{remark}
One can notice that the classification error is defined thanks to the expectation $E_{s \sim \mu_E}$. In the classical framework of classification, the data $\{s_i,a_i=\pi_E(s_i)\}_{1\leq i \leq D}$ are generated independently according to a distribution $\mu_{\text{Data}}$ over $S$ and hence the classical classification error must be:
\begin{equation}
\epsilon_C=E_{s \sim \mu_{\text{Data}}}[\mathds{1}_{\{\pi_C(s)\neq\pi_E(s)\}}].
\end{equation}
Then one can suppose that our data $\{s_i,a_i=\pi_E(s_i)\}_{1\leq i \leq D}$ are generated independently thanks to $\mu_E$ but this assumption is quite strong and non-realistic.
However it is often the case that the data provided by the expert are trajectories. In that case the theorem \ref{theoreme : mixing exponential}, which says that the rate of convergence
to the stationary distribution is at least exponential, allows us to suppose that the data $\{s_i,a_i=\pi_E(s_i)\}_{1\leq i \leq D}$ are generated under $\mu_E$.
\end{remark}
\begin{theorem}
\label{theorem : results}
Let $\{S,A,P,\gamma\}$ be a finite MDP without reward and $\pi_E$ an expert policy.
The notations $q$, $\pi_C$, $\hat{\pi}_C$, $\hat{R}^C$ are introduced in the Section \ref{section: Cascading}.
If $P_{\pi_E}$ is reducible, then $\mu_E$ is the stationary distribution of $\pi_E$ and:
\begin{enumerate}
\item $\pi_C$ is optimal for the reward $R^C$.
\item $\mu_E^T(V^{\pi_C}_{R^C}-V^{\pi_E}_{R^C})\leq\frac{\epsilon_C\max_{s\in S}(q(s,\pi_C(s))-\min_{a\in A}q(s,a))}{1-\gamma}$.
\item $\mu_E^T(V^{\hat{\pi}_C}_{\hat{R}^C}-V^{\pi_E}_{\hat{R}^C})\leq \frac{\epsilon_C\max_{s\in S}(q(s,\pi_C(s))-\min_{a\in A}q(s,a))+\|\epsilon^C_R\|_{\infty}+\mu_E^T\epsilon^E_R}{1-\gamma}$.
\end{enumerate}
\end{theorem}
In order to understand why this theorem is useful, let us make some important assumptions. Let us suppose that the classification and the regression steps are perfect in the sense that $\epsilon_R(s,a)=0,\forall (s,a)\in S\times A$ and $\epsilon_C=0$. Then we obviously have, thanks to the theorem \ref{theorem : results}, that $\pi_E=\pi_C$ is optimal for $\hat{R}^C=R^C$. Thus the method is able to provide a non-trivial reward function for which the policy $\pi_E$ is optimal. Moreover if the classification step and the regression step are not perfect, the theorem \ref{theorem : results} shows, that our approach is able to provide a non trivial-reward $\hat{R}^C$ for which the policy $\pi_E$ is near-optimal in the sense that:
\begin{equation}
E_{s \sim \mu_E}[V^{\hat{\pi}_C}_{\hat{R}^C}(s)-V^{\pi_E}_{\hat{R}^C}(s)]\leq B[\epsilon_C+\|\epsilon^C_R\|_{\infty}+\mu_E^T\epsilon^E_R], B\in\mathbb{R}_+.
\end{equation}
\begin{remark}
It is important to be clear about this result. If the only available data are provided by the equation \eqref{equation:data}, it is possible to control
$\epsilon_C$ and $\mu_E^T\epsilon^E_R$ because these errors depend only on the expert policy $\pi_E$. However it is not possible to control the error $\|\epsilon^C_R\|_{\infty}$
because it depends on the policy $\hat{\pi}_C$ which can be different than the expert policy and hence do not appear in the available data \eqref{equation:data}. However
it will be possible to obtain a control on the term $\|\epsilon^C_R\|_{\infty}$ if the data used for the regression are:
\begin{equation}
D_R=\{(s_{R,i},a_{R,i},s'_{R,i})\}_{1\leq i \leq D'},
\end{equation}
where $(s_{R,i},a_{R,i})$ are uniformly chosen on the set $S\times A$ or sampled from other policies than the expert. So, theoretically an easy way to be sure
to control the error $\|\epsilon^C_R\|_{\infty}$ is to be able to give a data set for the regression which is sampled from the expert policy and other policies (and more particularly $\hat{\pi}_C$). But we give examples, see Section \ref{section: experiments}, where the regression data set given by the equation \eqref{equation:data} is sufficient to obtain good results.
A possible argument to explain the fact that classification-regression still works when $D_C=D_R$, is that $\hat{\pi}_C$ must not be so different than $\pi_E$ but we did not manage to control
the term $\|\epsilon^C_R\|_{\infty}$ when $D_C=D_R$.
\end{remark}
\subsection{Proofs}

\begin{proof}[Lemma \ref{lemma: calculs V}]
Here, we use \eqref{equation: stationarity}
\begin{align}
\mu_\pi^TV^\pi_R&=\mu_\pi^T(R_\pi+\gamma P_\pi V^\pi_R)=\mu_\pi^TR_\pi+ \gamma\mu_\pi^TP_\pi V^\pi_R,
\\
&=\mu_\pi^TR_\pi+ \gamma\mu_\pi^TV^\pi_R=\frac{1}{1-\gamma}\mu_\pi^TR_\pi.
\end{align}
\end{proof}
\begin{proof}[Theorem \ref{theorem : results}]
In order to prove the three results of the theorem \ref{theorem : results}, let us introduce the function $R_E$ from $S\times A$ to $\mathbb{R}$ such that:
\begin{equation}
R^E(s,a)=q(s,a)-\gamma\sum_{s'\in S}p(s,a,s')q(s',\pi_E(s')), \forall a \in A, \forall s\in S.
\end{equation}
The first step is to show that:
\begin{align}
&q(s,\pi_C(s))=V^{\pi_C}_{R^C}(s), \forall s\in S,
\\
&q(s,\pi_E(s))=V^{\pi_E}_{R^E}(s), \forall s\in S.
\end{align}
This is quite straightforward because the column vector $q_{\pi_E}=[q(s,\pi_E(s))]_{1\leq i\leq N}$ is the fixed point of the operator $T^{\pi_E}_{R^E}$ and  $q_{\pi_C}=[q(s,\pi_C(s))]_{1\leq i\leq N}$ is the fixed point of the operator $T^{\pi_C}_{R^C}$:
\begin{align}
T^{\pi_E}_{R^E}(q_{\pi_E})&=R^E_{\pi_E}+\gamma P_{\pi_E}q_{\pi_E},
\\
&=q_{\pi_E}-\gamma P_{\pi_E}q_{\pi_E}+\gamma P_{\pi_E}q_{\pi_E}=q_{\pi_E},
\\
T^{\pi_C}_{R^C}(q_{\pi_C})&=R^C_{\pi_C}+\gamma P_{\pi_C}q_{\pi_C},
\\
&=q_{\pi_C}-\gamma P_{\pi_C}q_{\pi_C}+\gamma P_{\pi_C}q_{\pi_C}=q_{\pi_C}.
\end{align}
Moreover it is clear that $q_a=[q(s_i,a)]_{1\leq i\leq N}=Q^{\pi_C}_{R^C,a}=[Q^{\pi_C}_{R^C}(s_i,a)]_{1\leq i\leq N}$ for all $a \in A$:
\begin{align}
Q^{\pi_C}_{R^C,a}&=R^C_a+\gamma P_a V^{\pi_C}_{R^C}, \forall a\in A,
\\
&=R^C_a+\gamma P_a q_{\pi_C}=q_a-\gamma P_a q_{\pi_C} + \gamma P_a q_{\pi_C}=q_a, \forall a\in A.
\end{align}
So $q(s,a)=Q^{\pi_C}_{R^C}(s,a),\forall s\in S,\forall a\in A$ and as:
\begin{equation}
\pi_C(s)\in\argmax_{a\in A}q(s,a), \forall s\in S,
\end{equation}
$\pi_C$ is optimal for the reward $R^C$.
Now let us prove that:
\begin{equation}
\mu_E^TV^{\pi_C}_{R^C}-\mu_E^TV^{\pi_E}_{R^C}\leq\frac{\epsilon_C\max_{s\in S}(q(s,\pi_C(s))-\min_{a\in A}q(s,a))}{1-\gamma}.
\end{equation}
Indeed:
\begin{equation}
\mu_E^T(V^{\pi_C}_{R^C}-V^{\pi_E}_{R^C})=\mu_E^T(V^{\pi_C}_{R^C}-V^{\pi_E}_{R^E}+V^{\pi_E}_{R^E}-V^{\pi_E}_{R^C}).
\end{equation}
And $\mu_E^T(V^{\pi_C}_{R^C}-V^{\pi_E}_{R^E})$ is such that:
\begin{align}
\mu_E^T(V^{\pi_C}_{R^C}-V^{\pi_E}_{R^E})&=\sum_{s\in S}\mu_E(s)[V^{\pi_C}_{R^C}(s)-V^{\pi_E}_{R^E}(s)],
\\
&=\sum_{s\in S}\mu_E(s)[q(s,\pi_C(s))-q(s,\pi_E(s))]
\\
&=\sum_{s\in S}\mu_E(s)[q(s,\pi_C(s))-q(s,\pi_E(s))]\mathds{1}_{\{\pi_C(s)\neq\pi_E(s)\}},
\\
&\leq\epsilon_C\max_{s\in S}(q(s,\pi_C(s))-q(s,\pi_E(s))),
\\
&\leq\epsilon_C\max_{s\in S}(q(s,\pi_C(s))-\min_{a\in A}q(s,a)).
\end{align}
It remains to deal with the term $\mu_E^T(V^{\pi_E}_{R^E}-V^{\pi_E}_{R^C})$ using the fact that $\mu_E^TP_{\pi_E}=\mu_E^T$ and the lemma \ref{lemma: calculs V}:
\begin{align}
\mu_E^T(V^{\pi_E}_{R^E}-V^{\pi_E}_{R^C})&=\frac{1}{1-\gamma}\mu_E^T(R^E_{\pi_E}-R^C_{\pi_E}),
\\
&=\frac{1}{1-\gamma}\mu_E^T(\gamma P_{\pi_E}q_{\pi_C}-\gamma P_{\pi_E}q_{\pi_E}),
\\
&=\frac{\gamma}{1-\gamma}\mu_E^T(q_{\pi_C}-q_{\pi_E}),
\\
&=\frac{\gamma}{1-\gamma}\sum_{s\in S}\mu_E(s)[q(s,\pi_C(s))-q(s,\pi_E(s))]\mathds{1}_{\{\pi_C(s)\neq\pi_E(s)\}},
\\
&\leq\frac{\gamma}{1-\gamma}\epsilon_C\max_{s\in S}(q(s,\pi_C(s))-q(s,\pi_E(s))),
\\
&\leq\frac{\gamma}{1-\gamma}\epsilon_C\max_{s\in S}(q(s,\pi_C(s))-\min_{a\in A}q(s,a)).
\end{align}
Finally:
\begin{align}
\mu_E^T(V^{\pi_C}_{R^C}-V^{\pi_E}_{R^C})&=\mu_E^T(V^{\pi_C}_{R^C}-V^{\pi_E}_{R^E}+V^{\pi_E}_{R^E}-V^{\pi_E}_{R^C}),
\\
&\leq\frac{(\gamma+1-\gamma)\epsilon_C\max_{s\in S}(q(s,\pi_C(s))-\min_{a\in A}q(s,a))}{1-\gamma},
\\
&=\frac{\epsilon_C\max_{s\in S}(q(s,\pi_C(s))-\min_{a\in A}q(s,a))}{1-\gamma}.
\end{align}
In order to finish the proof it remains to show that:
\begin{equation}
\mu_E^TV^{\hat{\pi}_C}_{\hat{R}^C}-\mu_E^TV^{\pi_E}_{\hat{R}^C}\leq \frac{\epsilon_C\max_{s\in S}(q(s,\pi_C(s))-\min_{a\in A}q(s,a))+\|\epsilon^C_R\|_{\infty}+\|\epsilon^E_R\|_{\infty}}{1-\gamma}.
\end{equation}
We notice that:
\begin{equation}
\mu_E^T(V^{\hat{\pi}_C}_{\hat{R}^C}-V^{\pi_E}_{\hat{R}^C})=\mu_E^T(V^{\hat{\pi}_C}_{\hat{R}^C}-V^{\hat{\pi}_C}_{R^C}+V^{\hat{\pi}_C}_{R^C}-V^{\pi_E}_{R^C}+V^{\pi_E}_{R^C}-V^{\pi_E}_{\hat{R}^C}).
\end{equation}
It is very easy to see that:
\begin{align}
&\|V^{\hat{\pi}_C}_{\hat{R}^C}-V^{\hat{\pi}_C}_{R^C}\|_{+\infty}\leq\frac{\|\epsilon^C_R\|_{\infty}}{1-\gamma},
\\
&\mu_E^T(V^{\pi_E}_{R^C}-V^{\pi_E}_{\hat{R}^C})\leq\frac{\mu_E^T\epsilon^E_R}{1-\gamma},
\end{align}
Indeed:
\begin{align}
V^{\hat{\pi}_C}_{\hat{R}^C}(s)-V^{\hat{\pi}_C}_{R^C}(s)&=E^{\hat{\pi}_C}_s[\sum_{t=0}^{+\infty}\gamma^t(\hat{R}^C(s_t,\hat{\pi}_C(s_t))-R^C(s_t,\hat{\pi}_C(s_t)))], \forall s\in S,
\\
&\leq\frac{\|\epsilon^C_R\|_{\infty}}{1-\gamma}, \forall s\in S.
\end{align}
And, thanks to the lemma \ref{lemma: calculs V}:
\begin{align}
\mu_E^T(V^{\pi_E}_{R^C}-V^{\pi_E}_{\hat{R}^C})&=\mu_E^T(V^{\pi_E}_{\epsilon_R}),
\\
&\leq\frac{\mu_E^T\epsilon^E_R}{1-\gamma}.
\end{align}
Thus:
\begin{equation}
\label{equation: cas3-1}
\mu_E^T(V^{\hat{\pi}_C}_{\hat{R}^C}-V^{\hat{\pi}_C}_{R^C}+V^{\pi_E}_{R^C}-V^{\pi_E}_{\hat{R}^C})\leq\frac{\mu_E^T\epsilon^E_R+\|\epsilon^C_R\|_{\infty}}{1-\gamma}.
\end{equation}
It remains to deal with the term $\mu_E^T(V^{\hat{\pi}_C}_{R^C}-V^{\pi_E}_{R^C})$:
\begin{equation}
\mu_E^T(V^{\hat{\pi}_C}_{R^C}-V^{\pi_E}_{R^C})=\mu_E^T(V^{\hat{\pi}_C}_{R^C}-V^{\pi_C}_{R^C}+V^{\pi_C}_{R^C}-V^{\pi_E}_{R^C}).
\end{equation}
As $\pi_C$ is optimal for the reward $R^C$ then:
\begin{equation}
\mu_E^T(V^{\hat{\pi}_C}_{R^C}-V^{\pi_C}_{R^C})\leq 0.
\end{equation}
So:
\begin{equation}
\label{equation: cas3-2}
\mu_E^T(V^{\hat{\pi}_C}_{R^C}-V^{\pi_E}_{R^C})\leq \mu_E^T(V^{\pi_C}_{R^C}-V^{\pi_E}_{R^C})\leq \frac{\epsilon_C\max_{s\in S}(q(s,\pi_C(s))-\min_{a\in A}q(s,a))}{1-\gamma}.
\end{equation}
Finally by regrouping the results in \eqref{equation: cas3-1} and \eqref{equation: cas3-2}:
\begin{equation}
\mu_E^TV^{\hat{\pi}_C}_{\hat{R}^C}-\mu_E^TV^{\pi_E}_{\hat{R}^C}\leq \frac{\epsilon_C\max_{s\in S}(q(s,\pi_C(s))-\min_{a\in A}q(s,a))+\|\epsilon^C_R\|_{\infty}+\mu_E^T\epsilon^E_R}{1-\gamma}.
\end{equation}
\end{proof}

\section{Experiments}
\label{section: experiments}
\subsection{Instantiation of the algorithm}
As hinted to in Section \ref{section: Cascading}, our algorithm can be instanciated in many ways, as any off-the-shelf classifier using a score function can be used in the first step, and any regression method can be used in the second step. This freedom of choice leaves room for problem dependant tweaks if needed.\\

In our tests, we used a subgradient-descent based classifier and a simple regularized least square regression method. The classifier tries to minimize the risk function $R_N(q)$ :
\begin{equation}
  R_N(q) = {1\over N} \sum_{i=1}^N\left(\max_{a}(q(s_i,a) + l(s_i,a)) - q(s_i,a_i) \right).
\end{equation}
where $l$ is a user defined function whose role is to quantify the gap between the best and second to best actions. In all our tests we used the naive solution $l(s_i,a) = 1$ if $a\neq a_i$ and $l(s_i,a_i) = 0$. To minimize the risk, we introduce the parametrization 
\begin{equation}
  q(s,a) = \omega^T\phi(s,a).
\end{equation}
The risk now becomes :
\begin{equation}
  R_N(\omega) = {1\over N} \sum_{i=1}^N\left(\max_{a}(\omega^T\phi(s_i,a) + l(s_i,a)) - \omega^T\phi(s_i,a_i) \right).
\end{equation}

Because of the $\max$ in the expression, we can not use a simple gradiend descent on $\omega$ to minimize the risk. We have to make use of the subgradient. Let $a^* = \max_a\omega^T\phi(s,a)$, we get the following update rule :
\begin{equation}
  \omega_{t+1} = \omega_t -\alpha_t{1\over N}\sum_{i=1}^N\left(\phi(s_i,a^*) - \phi(s_i,a_i)\right).
\end{equation}
The regressor is a classical least square fitting. Given some features $\psi : S\rightarrow \mathbb{R}^p$, we define :
\begin{eqnarray}
\Psi &= \begin{pmatrix}\psi(s_1)^T\\\vdots\\\psi(s_N)^T\end{pmatrix}\\
\textrm{and}
\hat R &= \begin{pmatrix}\hat r_1 \\\vdots\\\hat r_N\end{pmatrix}\\
\end{eqnarray}
with $\hat r_i$ as in equation \ref{ri.def}. We then get :
\begin{equation}
\theta = (\Psi^T\Psi + \lambda Id)^{-1}\Psi^T\hat R,
\end{equation}
so that $\hat R^C(s) = \theta^T \psi(s)$.\\
\subsection{Experimental setting}
For both problems we test our approach with, we compare it to two baselines : the mean performance of an agent trained on a random reward and the mean performance of the policy output by the algorithm of \cite{abbeel2004apprenticeship}. This algorithm can be instanciated to work in the same dire conditions as our algorithm, that is to say using only samples from the expert. To do this, we estimate the feature expectations and solve the MDP in a batch, off policy manner thanks respectively to to LSTD$\mu$ \cite{klein2011batch} and LSPI \cite{lagoudakis2003least}. When this does not work, we fall back to a setting where everything is known, that is to say the feature expectations are computed by a dynamic programming algorithm that make use of the transition probabilities.\\
The performance of the expert and the agents is computed with respect to the reward the expert was trained with. We use as a performance metric the mean value over the state space.
\subsection{Gridworld}
We first tested our approach on a simple $5\times 5$ gridworld. The expert and the agent can choose between 4 actions (down, left right, or up) that sends the player in the corresponding neighbooring cell. Trying to go off the grid will result in the player staying in its position. Each action has a probability $0.3$ of failing, in which case another randomly chosen different action is executed. The expert begins in the lower left corner and is trained to go to the higher right corner by an exact dynamic programming algorithm. The original reward is null everywhere but in the higher right corner where it is $1$.\\
The results are shown in two figures. Figure \ref{Fig1.fig} show the baselines we compare ourselves with. Pure classification and the algorithm of \cite{abbeel2004apprenticeship} are on par with one another, both far better than training an agent on a random reward ; they both converge to a value close to the expert's given enough samples. Figure \ref{Fig2.fig} shows results of our algorithm together with the mean performance of both forementionned approaches (variance and min/max were dropped for legibility's sake). It shows the soundness of our cascading approach : when only a few samples are available, we provide a reward that lead to a better policy. When samples are widely available all approaches converge to the same value.
  \begin{figure}
  \begin{tabular}{ccc}
  \subfigure[State of the art approaches on the GridWorld]{\includegraphics[width=.45\linewidth]{"Fig1"}\label{Fig1.fig}}&\hspace{.05\linewidth}&
  \subfigure[Our new approach on the GridWorld]{\includegraphics[width=.45\linewidth]{"Fig2"}\label{Fig2.fig}}\\
  \subfigure[All approaches on the Highway driving simulator]{\includegraphics[width=.4\linewidth]{"Fig3"}\label{Fig3.fig}}&\hspace{.05\linewidth}&\includegraphics[width=0.45\linewidth]{"Legend"}
  \end{tabular}
  \caption{Results on the GridWorld and Highway problems. Data is shown with mean, standard deviation, minimum and maximum value over 50 runs.}
  \end{figure}



\subsection{Highway driving simulator}
There are problems for which pure classification just does not cut it, as the number of samples that would be needed to imitate the expert's control is dauntingly large. To illustrate this, inspired from a problem seen in \cite{syed2008apprenticeship,syed2008game}, we coded a driving simulator. The agent controls a car that can switch between the three lanes of the road, go off-road on either side and modulate between three speed levels. The first thing to note about this setting is that the direct problem (RL) problem is not trivial, as LSPI is unable to solve it using the natural features we first gave it. These feature stemmed from a discretization of the state-action space, the dimension of the features being the cardinality of this space. With each state-action pair $s,a$ we associate an index $i$ so that the $i$-th coordinate of the feature is $1$ and every other is $0$. We tried to train an agent on the reward used by the expert (which was trained using an exact dynamic programming approach), but on top of being extremely slow to run (the LSTD$Q$ subroutine implies a matrix inversion) it does not converge.\\
As LSPI is used in our instanciation of \cite{abbeel2004apprenticeship}'s algorithm to solve the MDP, it is impossible to use the latter with the natural features. Although it proved possible to engineer features so that LSPI converges thus making the state of the art IRL algorithm work, this is outside the scope of this document. As a baseline we used an instanciation of \cite{abbeel2004apprenticeship}'s algorithm that imake use of all the information about the problem. It is displayed as an horizontal line in figure \ref{Fig3.fig} because it does not use samples from the expert, instead it uses the transition probabilities and the expert's policy to compute the exact feature expectation of the expert and every intermediate policy.\\
As predicted, pure classification is not sample efficient enough to provide a good control whereas our approach is able to generalize from very little data and exhibits performance on par with the \emph{fully-informed} state of the art algorithm when given as few as 100 samples.\\
Furthermore, the state of the art algorihtm make use of intermediate policies, which is computationally costly particularily when the state space gets big. Our approach computational cost scales with the number of samples we are given, which is as we demonstrated a small number. Experimentally, when producing the data for our graph, it took less time to run the cascading approach 50 times for 6 abcissa than to run \cite{abbeel2004apprenticeship}'s algorithm once.\\
\section{Related Work}
\label{section: related work}
All the IRL approaches summed up in \cite{neu2009training}, namely the historical approach of \cite{ng2000algorithms} as well as for example \cite{abbeel2004apprenticeship,syed2008apprenticeship,syed2008game, ziebart2008maximum} share the same pivotal argument of trying to minimize some distance between the feature expectation of the expert and the feature expectation of another policy through successive updates of this intermediate policy. The main problem with these algorithm is that the update on the policy is done through the update of a reward function. This means that the forward problem of RL must be repeatedly solved ; this is often non trivial, and almost always computationally costly.\\

Our algorithm avoid this pitfall, specifically, one needs more information to solve the forward problem than to just compute the feature expectation of the expert as our algorithm does. Even a recent contribution like \cite{boularias2011relative} still needs samples that cover a large portion of the state space although it does not explicitely require the ability to solve the forward problem.\\

A recurring argument is to see the expert not only as optimal but as optimal by a certain margin. The action chosen by the expert should have a quality strictly greater than any other action. Structured planning as done in \cite{ratliff2006maximum,ratliff2007imitation, ratliff2007boosting, kolter2008hierarchical} allow some influence on the margin. Through the $l$ function of our classifier, we share the same feature.\\

Finally, \cite{ratliff2007imitation} cast the imitation learnig problem as a multi class classification problem. As seen in the experimental results, the power of generalization of our approach is greater than pure classification thanks to the regression step. This allows us to be more sample efficient than any other published approach known to us.
\section{Conclusion}
We provide a new solution to the IRL problem that avoid most pitfalls of existing approaches. It outputs a reward function, the most succinct and transferrable description of a task without having to repeatedly solve the forward RL problem. The only information our algorithm need is provided by samples from the expert, which is to the best of our knowledge a feature that can not be found in any other published algorithm. Even recent other approaches need randomly drawn samples. We provided theoretical results showing the soundness of our approach as well as experimental benchmarks comparing it to a widely known state of the art algorithm. In a near furure, experiments on real world robotics problems will take place. We hope to benchmark our algorithm against the few other approaches shown to work in such a difficult context.

\bibliographystyle{alpha}
\bibliography{Biblio}
\end{document}
