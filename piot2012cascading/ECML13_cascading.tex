% Created 2012-10-15 lun. 14:49
\documentclass[smallextended]{svjour3}
\smartqed 
\usepackage[utf8]{inputenc}
\usepackage[T1]{fontenc}
\usepackage{fixltx2e}
\usepackage{graphicx}
\usepackage{longtable}
\usepackage{float}
\usepackage{wrapfig}
\usepackage{soul}
\usepackage{textcomp}
\usepackage{marvosym}
\usepackage[integrals]{wasysym}
\usepackage{latexsym}
\usepackage{amssymb}
\usepackage{hyperref}
\usepackage{algorithm}
\usepackage{algorithmic}
\usepackage{subfigure}
\tolerance=1000
%\usepackage{amsmath}
%\usepackage{amsthm}
%\newtheorem{definition}{Definition}
%\newtheorem{theorem}{Theorem}
\newtheorem{cor}{Corollary}
\usepackage{dsfont}
\usepackage{mathtools}
\mathtoolsset{showonlyrefs=true}
\providecommand{\alert}[1]{\textbf{#1}}
\newcommand{\E}{\mathbf{E}}
\newcommand{\argmax}{\operatorname*{argmax}} %\operatorname* pour les op. pouvant admettre des limites...
\newcommand{\Card}{\operatorname*{Card}} %\operatorname* pour les op. pouvant admettre des limites...
\begin{document}
\title{A cascaded supervised learning approach to inverse reinforcement learning}
\author{Edouard Klein$^{1,2}$ \and Bilal Piot $^{2,3}$\and Matthieu Geist $^{2}$\and Olivier Pietquin$^{2,3}$}
\titlerunning{Cascading approach to IRL}
\institute{
 1. ABC Team\\
 LORIA-CNRS, France.\\
\and
2. Supélec\\
 IMS-MaLIS Research group, France\\
 \texttt{firstname.lastname@supelec.fr}\\
\and 
3. UMI 2958 (GeorgiaTech-CNRS)\\
France\\
}
\date{\today}


\maketitle

\begin{abstract}
  This paper considers the Inverse Reinforcement Learning (IRL) problem, that is inferring a reward function for which a demonstrated expert policy is optimal.
We propose to break the IRL problem down into two generic Supervised Learning steps: this is the Cascaded Supervised IRL (CSI) approach. A classification step that defines a score function is followed by a regression step providing a reward function.
A theoretical analysis shows that the demonstrated expert policy is near-optimal for the computed reward function.
Not needing to repeatedly solve a Markov Decision Process (MDP) and the ability to leverage existing techniques for classification and regression are two important advantages of the CSI approach. It is furthermore empirically demonstrated to work with only transitions sampled according to the expert policy, up to the use of some heuristics. This is exemplified on two classical benchmarks (a highway driving simulator and an inverted pendulum simulator).
  \end{abstract}
\section{Introduction}
\label{sec-2}
Sequential decision making consists in choosing the appropriate action given the available data in order to maximize a certain criterion. When framed in a Markov Decision Process (MDP) (see Sec.~\ref{sec:background}), (Approximate) Dynamic programming ((A)DP) or Reinforcement Learning (RL) are often used to solve the problem by maximizing the expected sum of discounted rewards. The Inverse Reinforcement Learning (IRL)~\cite{russell1998learning} problem, which is addressed here, aims at inferring a reward function for which a demonstrated expert policy is optimal.

IRL is one of many ways to perform Apprenticeship Learning (AL): imitating a demonstrated expert policy, without explicitely looking for the reward function. The reward function nevertheless is of interest in its own right. As mentionned in \cite{russell1998learning}, its semantics can be analyzed in biology or econometrics for instance. Practically, the reward can be seen as a succinct description of a task. Discovering it removes the coupling between understanding the task and learning how to fulfill it that exists in AL. IRL allows the use of (A)DP or RL techniques to learn the task from the computed reward function. Although some approaches use it as a byproduct, most AL algorithm makes no use of the reward function, or have nothing to do with IRL altogether. A very straightforward non-IRL way to do AL is for example to use a multi-class classifier to directly learn the expert policy. We provide in the experiments (Sec.~\ref{sec:experiments}) a comparison between AL and IRL algorithms by using IRL as a way to do AL.

A lot of existing approaches in either IRL or IRL-based AL need to repeatedly solve the underlying MDP to find the optimal policies of intermediate reward functions. Thus, their performance depends strongly on the quality of the associated subroutine. Consequently, they suffer from the same challenges of scalability, data scarcity, etc., as RL and (A)DP. In order to avoid \emph{repeatedly} solving such problems, we adopt a different point of view.

Having in mind that there is a one to one relation between a reward function and its associated optimal action-value function (via the Bellman equation, see Eq.~\eqref{eq:bellman1}), it is worth thinking of a method able to output an action-value function for which the greedy policy is the demonstrated expert policy. Thus, the demonstrated expert policy will be optimal for the related reward function. We propose to use a score function-based multi-class classification step (see Sec.~\ref{sec:algo}) to infer a score function. Besides, in order to retrieve via the Bellman equation the reward associated to the score function computed by the classification step, we introduce a regression step (see Sec.~\ref{sec:algo}). That is why the method is called the Cascaded Supervised Inverse reinforcement learning (CSI). This method is analyzed in Sec.~\ref{sec:analysis}, where it is shown that the demonstrated expert policy is near-optimal for the reward the regression step outputs.

This algorithm does not need to iteratively solve an MDP and needs only sampled transitions from expert and non-experts policies as inputs. Moreover, up to the use of some heuristics (see Sec.~\ref{subsubsec:heuristics}), the algorithm is able to be trained only with transitions sampled from the demonstrated expert policy. A specific instantiation of CSI (proposed Sec.~\ref{subsec:instanciation}) is tested on an inverted pendulum simulator (Sec.~\ref{subsec:IP}) and a car driving simulator (Sec.~\ref{subsec:highway}) where we compare it with an AL algorithm \cite{abbeel2004apprenticeship}, with a pure classification algorithm \cite{taskar2005learning} and with a recent successfull IRL method \cite{klein2012scirl} as well as with a random baseline.

Differences and similarities with existing AL or IRL approaches are succinctly exposed in Sec.~\ref{sec:related}.
\section{Background and notations}
\label{sec:background}
First, we introduce some general notations.
Let $E$ and $F$ be two non-empty sets, $E^F$ is the set of functions from $F$ to $E$.
We note $\Delta_X$ the set of distributions over $X$.
Let $\alpha\in\mathbb{R}^X$ and $\beta\in\mathbb{R}^X$: $\alpha\geq\beta \Leftrightarrow \forall x\in X, \alpha(x) \geq \beta(x)$. We will often slightly abuse the notations and consider (where applicable) most objects as if they were matrices and vectors indexed by the set they operate upon.

We work with finite MDPs \cite{puterman1994markov}, that is tuples $\{S,A,P,R,\gamma\}$. The state space is noted $S$, $A$ is a finite action space, $R\in\mathbb{R}^{S\times A}$ is a reward-function, $\gamma\in ]0,1[$ is a discount factor and $P\in \Delta_{S}^{S\times A}$ is the Markovian dynamics of the MDP. Thus, for each $(s,a)\in S\times A$, $P(.|s,a)$ is a distribution over $S$ and $P(s'|s,a)$ is the probability to reach $s'$ by choosing action $a$ in state $s$. At each time step $t$, the agent uses the information encoded in the state $s_t\in S$ in order to choose an action $a_t \in A$ according to a (deterministic\footnote{We restrict ourselves here to deterministic policies, but the loss of generality is minimal as there exists at least one optimal deterministic policy.}) policy $\pi\in A^S$. The agent then steps to a new state $s_{t+1}\in S$ according to the Markovian transition probabilities $P(s_{t+1}|s_t,a_t)$. The stationary distribution over the states $\rho_\pi$ induced by a policy $\pi$ satisfies
\begin{equation}
  \rho_\pi^TP_\pi = \rho_\pi^T,
  \end{equation}
where $P_\pi$ is $(P(s'|s,\pi(s)))_{s,s' \in S}$. The stationary distribution relative to the exert policy $\pi_E$ is $\rho_E$.

The reward function $R$ is a local measure of the quality of the control. The global quality of the control induced by a policy $\pi$, with respect to a reward $R$, is assessed by the value function $V^\pi_R \in \mathbb{R}^{S}$ which associates to each state the expected discounted cumulative reward for following policy $\pi$ from this state:
\begin{equation}
V^\pi_R(s) = \E[\sum_{t\geq 0}\gamma^tR(s_t,\pi(s_t))|s_0 = s].
\end{equation}
This long-term criterion is what is being optimized when solving an MDP. Therefore, an optimal policy $\pi^*_R$ is a policy whose value function (the optimal value function $V^*_R$) is greater than that of any other policy, for all states: $\forall \pi, V^*_R\geq V^\pi_R$.

The Bellman evaluation operator $T^\pi_R: \mathbb{R}^{S} \rightarrow  \mathbb{R}^{S}$ is defined by
\begin{equation}
  T^{\pi}_RV = R_\pi + \gamma P_\pi V
  \end{equation}
where $R_\pi$ is $(R(s,\pi(s)))_{s\in S}$. The Bellman optimality operator follows naturally:
\begin{equation}
  T^*_RV = \max_\pi T^\pi_RV.
\end{equation}
  Both operators are contractions. The fixed point of the Bellman evaluation operator $T^\pi_R$ is the value function of $\pi$ with respect to reward R:
  \begin{eqnarray}
    V^\pi_R &=& T^\pi_R V^\pi_R \Leftrightarrow V^\pi_R = R_\pi + \gamma P_\pi V^\pi_R.
  \end{eqnarray}
The Bellman optimality operator $T^*_R$ also admits a fixed point, the optimal value function $V_R^*$ with respect to reward $R$.

Another object of interest is the action-value function $Q^\pi_R\in\mathbb{R}^{S\times A}$ that adds a degree of freedom on the choice of the first action, formally defined by $Q^\pi_R(s,a) = T^a_RV^\pi_R(s)$, with $a$ the policy that always returns action $a$ ($T^a_RV = R_a + \gamma P_a V$ with $P_a = (P(s'|s,a))_{s,s' \in S}$ and $R_a$ = $(R(s,a))_{s\in S}$). The value function $V^\pi_R$ and the action-value function $Q^\pi_R$ are quite directly related: $\forall s \in S, V^\pi_R(s) = Q^\pi_R(s,\pi(s))$. The Bellman evaluation equation for $Q^\pi_R$ is therefore:
\begin{equation}
  Q^\pi_R(s,a) = R(s,a) + \gamma \sum_{s'\in S}P(s'|s,a) Q(s',\pi(s')).
  \label{eq:bellman1}
\end{equation}

An optimal policy follows a greedy mechanism with respect to its optimal action-value function $Q^*_R$:
\begin{equation}
  \label{eq:greedy}
  \pi^*_R(s)\in\argmax_aQ^*_R(s,a).
\end{equation}

When the state space is too large to allow matricial representations or when the transition probabilites or even the reward function are unknown except through observations gained by interacting with the system, RL or ADP may be used to approximate the optimal control policy~\cite{sutton1998reinforcement}.

We recall that solving the MDP is the direct problem. This contribution aims at solving the inverse one. We observe trajectories drawn from an expert's deterministic\footnotemark[\value{footnote}] policy $\pi_E$, assuming that there exists some unknown reward $R_E$ for which the expert is optimal. The suboptimality of the expert is an interesting setting that has been discussed for example in \cite{melo2010analysis,syed2010reduction}, but that we are not addressing here. We do not try to find this unknown reward $R_E$ but rather a non trivial reward $R$ for which the expert is at least near-optimal. The trivial reward $0$ is a solution to this ill-posed problem (no reward means that every behavior is optimal). Because of its ill-posed nature, this expression of \emph{Inverse Reinforcement Learning} (IRL) still has to find a satisfactory solution although lots of progress have been made, see Sec.~\ref{sec:related}.
\section{The cascading algorithm}
\label{sec:algo}
Our first step towards a reward function solving the IRL problem is a classification step using a \emph{score function-based multi-class classifier} (SFMC$^2$ for short). This classifier learns a score function $q\in\mathbb{R}^{S\times A}$ that rates the association of a given label $a\in A$ with a certain input $s\in S$. The classification rule $\pi_C\in A^S$ simply selects (one of) the label(s) that achieves the highest score for the given inputs:
\begin{equation}
  \label{eq:greedy2}
\pi_C(s) \in \argmax_a q(s,a).
\end{equation}
For example, \emph{Multi-class Support Vector Machines}~\cite{guermeur2011generic} can be seen as SFMC$^2$ algorithms, the same can be said of the structured margin approach \cite{taskar2005learning} which we consider in the experimental setting. Other algorithms may be envisionned (see Sec.~\ref{subsec:instanciation}). 

Given a dataset $D_C = \{(s_i,a_i=\pi_E(s_i))_i\}$ of actions $a_i$ (deterministically) chosen by the expert on states $s_i$, we train such a classifier. The classification policy $\pi_C$ is not the end product we are looking for (that would be mere imitation of the expert, not IRL). What is of particular interest to us is the score function $q$ itself. One can easily notice the similarity between Eq.~\eqref{eq:greedy2} and Eq.~\eqref{eq:greedy} that describes the relation between the optimal policy in an MDP and its optimal action-value function. The score function $q$ of the classifier can thus be viewed as some kind of optimal action-value function for the classifier policy $\pi_C$. By inversing the Bellman equation Eq.~\eqref{eq:bellman1} with $q$ in lieu of $Q^\pi_R$, one gets $R^C$, the reward function relative to our score/action-value function $q$:
\begin{equation}
  \label{eq:rc}
  R^C(s,a) =q(s,a) - \sum_{\mathrm{s'}}P(s'|s,a)q(s',\pi_C(s')).
\end{equation}
As we wish to solve the general, or approximate, IRL problem where the transition probabilities $P$ are unknown, our reward function $R^C$ will be approximated with the help of information gathered by interacting with the system. We assume that another dataset $D_R = \{(s_j,a_j,s'_j)_j\}$ is available where $s'_j$ is the state an agent taking action $a_j$ in state $s_j$ transitionned to. Action $a_j$ needs not be chosen by any particular policy. The dataset $D_R$ brings us information about the dynamics of the system. From it, we construct datapoints
\begin{equation}
  \label{eq:rj}
  \{\hat r_j = q(s_j,a_j) - \gamma q(s'_j,\pi_C(s'_j))\}_j.
\end{equation}
As $s'_j$ is sampled accordingly to $P(\cdot|s_j,a_j)$ the constructed datapoints help building a good approximation of $R^C(s=s_j,a=a_j)$. A regressor (a simple least-square approximator can do but other solutions such as Gaussian Processes \cite{rasmussen2006gaussian} could also be envisionned) is then fed the datapoints $((s_j,a_i),\hat r_j)$ to obtain $\hat R^C$, a generalisation of $\{((s_j,a_j),\hat r_j)_j\}$ over the whole state-action space. The complete algorithm is given in Alg.~\ref{algo:cascading}.
\begin{algorithm}%[H]
    %\small
  \caption{CSI algorithm}
  \label{algo:cascading}
  \emph{\textbf{Given}} a training set $D_C=\{(s_i,a_i=\pi_E(s_i))\}_{1\leq i \leq D}$ and another training set $D_R=\{(s_{j},a_{j},s'_{j})\}_{1\leq j \leq D'}$\;\\
  \emph{\textbf{Train}} a score function-based classifier on $D_C$, obtaining decision rule $\pi_C$ and score function $q:S\times A \rightarrow \mathbb R$\;\\
  \emph{\textbf{Learn}} a reward function $\hat R^C$ from the dataset $\{((s_{j},a_{j}),\hat{r}_j)\}_{1\leq j \leq D'}$, $\forall (s_j,a_j,s'_j) \in D_R,\hat{r}_j=q(s_{j},a_{j})-\gamma q(s'_{j},\pi_C(s'_{j}))$\;\\
  \emph{\textbf{Output}} the reward function $\hat R^{C}$ \;
\end{algorithm}

Cascading two supervised approaches like we do is a way to inject the MDP structure in the resolution of the problem. Indeed, mere classification only takes into account information from the expert (i.e., which action goes with which state) whereas using the Bellman equation in the expression of $\hat r_j$ makes use of the information lying in the transitions $(s_j,a_j,s'_j)$, namely information about the transition probabilities $P$. The final regression step is a way to generalize this information about $P$ to the whole state-action space in order to have a well-behaved reward function.

Being able to alleviate the ill effects of scalability or data scarcity by leveraging the wide range of techniques developped for the classification and regression problems is a strong advantage of the CSI approach.

\section{Analysis}
\label{sec:analysis}


In this section, we prove that the deterministic expert policy $\pi_E$ is near optimal for the reward $\hat{R}^C$ the regression step outputs.
More formally, recalling from Sec.~\ref{sec:background} that $\rho_E$ is the stationary distribution of the expert policy, we prove that $\E_{s\sim\rho_E}[V^*_{\hat{R}^C}(s)-V^{\pi_E}_{\hat{R}^C}(s)]$ is bounded by a term that depends on:
\begin{itemize}
  \item the classification error defined as
\begin{equation}  
\epsilon_C=\E_{s\sim\rho_E}[\mathds{1}_{\{\pi_C(s)\neq\pi_E(s)\}}];
\end{equation}
\item the regression error defined as 
  \begin{equation}
    \epsilon_R=\max_{\pi\in A^S}\|\epsilon_\pi^R\|_{1,\rho_E},
  \end{equation}
  with:
  \begin{itemize}
  \item the subscript notation already used for $R_\pi$ and $P_\pi$ in Sec.~\ref{sec:background} meaning that, given an $X\in\mathbb{R}^{S\times A}$, $\pi\in A^S$, and $a\in A$,  $X_\pi\in\mathbb{R}^S$ and $X_a\in\mathbb{R}^S$ are repectively such that: $\forall s\in S, X_\pi(s)=X(s,\pi(s))$ and $\forall s\in S, X_a(s)=X(s,a)$ ;
  \item $\epsilon^R_\pi=R^C_\pi-\hat{R}^C_\pi$ ;
  \item $\|.\|_{1,\mu}$ the $\mu$-weighted $L_1$ norm: $\|f\|_{1,\mu} = \E_{x\sim \mu}[|f(x)|]$ ;
  \end{itemize}
\item the concentration coefficient $C_* = C_{\hat \pi_C}$ with:
  \begin{itemize}
  \item $C_{\pi}=(1-\gamma)\sum_{t\geq0}\gamma^tc_{\pi}(t), \text{ with } c_{\pi}(t)=\max_{s\in S}\frac{(\rho_E^TP^t_\pi)(s)}{\rho_E(s)}$ ;
  \item $\hat \pi_C$, the optimal policy for the reward $\hat R^C$ output by the algorithm ;
  \end{itemize}
  The constant $C_*$ could be estimated \emph{a posteriori} (after $\hat{R}^C$ is computed). We could upper bound $C_*$ by a more usual and general concentration coefficient but $C_*$ gives a tighter final result (roughly, if $\hat{\pi}_C\approx\pi_E$ then $C_* \approx 1$).
\item $\Delta q$, defined by 
  \begin{eqnarray}
    \Delta q &= &\max_{s\in S}(\max_{a\in A}q(s,a)-\min_{a\in A}q(s,a))=\max_{s\in S}(q(s,\pi_C(s))-\min_{a\in A}q(s,a)),
    \end{eqnarray} which could be normalized to $1$ without loss of generality. The range of variation of $q$ induces the one of $R^C$, $\hat{R}^C$ and $V^{\pi_C}_{\hat{R^C}}$.
  \end{itemize}

\begin{theorem}
\label{thm}
Let $\pi_E$ be the deterministic expert policy, $\rho_E$ its stationary distribution and $\hat{R}^C$ the reward the cascading algorithm outputs. We have:
\begin{equation}
0\leq\E_{s\sim\rho_E}[V^*_{\hat{R}^C}(s)-V^{\pi_E}_{\hat{R}^C}(s)]\leq \frac{1}{1-\gamma}\left(\epsilon_C\Delta q +\epsilon_R(1+C_*)\right).
\end{equation}
\end{theorem}
\begin{proof}
First let's introduce some notations, $q\in\mathbb{R}^{S\times A}$ is the score function output by the classification step, $\pi_C$ is a deterministic classifier policy so that $\forall s \in S,\pi_C(s)\in\argmax_{a\in A}q(s,a)$. We recall the definition of the reward $R^C\in\mathbb{R}^{S\times A}$:
\begin{equation}
\forall (s,a)\in S\times A, R^C(s,a)=q(s,a) -\gamma\sum_{s'\in S}P(s'|s,a)q(s',\pi_C(s')),
\end{equation}
and the definition of $\hat{R}^C\in\mathbb{R}^{S\times A}$ which is the reward function output by the regression step.

The difference between $R^C$ and $\hat{R}^C$ is noted $\epsilon^R=R^C-\hat{R}^C$.
We also introduce the reward function $R^E\in\mathbb{R}^{S\times A}$ which will be useful in our proof:
\begin{equation}
\forall (s,a)\in S\times A, R^E(s,a)=q(s,a) -\gamma\sum_{s'\in S}P(s'|s,a)q(s',\pi_E(s')).
\end{equation}

We can now introduce the following vectorial equations:
\begin{align}
&R^C_a=q_a-\gamma P_aq_{\pi_C},
\\
&R^E_a=q_a-\gamma P_aq_{\pi_E},
\\
&\epsilon^R_a=R^C_a-\hat{R}^C_a.
\end{align}
Now, we are going to upper bound the term: $\E_{s\sim\rho_E}[V^*_{\hat{R}^C}-V^{\pi_E}_{\hat{R}^C}]\geq0$ (the lower bound is obvious as $V^*$ is optimal).
Recall that $\hat{\pi}_C$ is a deterministic optimal policy of the reward $\hat{R}^C$. First, the term $V^*_{\hat{R}^C}-V^{\pi_E}_{\hat{R}^C}$ is decomposed:
\begin{equation}
V^*_{\hat{R}^C}-V^{\pi_E}_{\hat{R}^C}=(V^{\hat{\pi}_C}_{\hat{R}^C}-V^{\hat{\pi}_C}_{R^C})+(V^{\hat{\pi}_C}_{R^C}-V^{\pi_E}_{R^C})+(V^{\pi_E}_{R^C}-V^{\pi_E}_{\hat{R}^C}).
\end{equation}
We are going to bound each of these three terms. First, let $\pi$ be a given deterministic policy. We have, using $\epsilon^R = R^C - \hat R^C$:
\begin{equation}
V^{\pi}_{R^C}-V^{\pi}_{\hat{R}^C}=V^{\pi}_{\epsilon^R}=(I-\gamma P_\pi)^{-1}\epsilon^R_{\pi}.
\end{equation}
If $\pi=\pi_E$, we have, thanks to the power series expression of $(I-\gamma P_{\pi_E})^{-1}$ and the definition of $\rho_E$:
\begin{equation}
  \rho_E^T(V^{\pi}_{R^C}-V^{\pi}_{\hat{R}^C})=\rho_E^T(I-\gamma P_{\pi_E})^{-1}\epsilon^R_{\pi_E}=\frac{1}{1-\gamma}\rho_E^T\epsilon^R_{\pi_E}\leq\frac{1}{1-\gamma}\|\epsilon^R_{\pi_E}\|_{1,\rho_E}.\end{equation}

If $\pi\neq\pi_E$, we use the concentration coefficient $C_\pi$. We have then:
\begin{equation}
  \rho_E^T(V^{\pi}_{R^C}-V^{\pi}_{\hat{R}^C})\leq\frac{C_\pi}{1-\gamma}\rho_E^T\epsilon^R_{\pi}\leq\frac{C_\pi}{1-\gamma}\|\epsilon^R_{\pi}\|_{1,\rho_E}.
  \end{equation}
So, using the notations introduced before we stated the theorem, we are able to give an upper bound to the first and third terms (recall also the notation $C_{\hat \pi_C} = C_*$):
\begin{equation}
\rho_E^T((V^{\pi_E}_{R^C}-V^{\pi_E}_{\hat{R}^C})+(V^{\hat{\pi}_C}_{\hat{R}^C}-V^{\hat{\pi}_C}_{R^C}))\leq\frac{1+C_*}{1-\gamma}\epsilon_R.
\end{equation}
Now, there is still an upper bound to find for the second term. It is possible to decompose it as follows:
\begin{equation}
V^{\hat{\pi}_C}_{R^C}-V^{\pi_E}_{R^C}=(V^{\hat{\pi}_C}_{R^C}-V^{\pi_C}_{R^C})+(V^{\pi_C}_{R^C}-V^{\pi_E}_{R^E})+(V^{\pi_E}_{R^E}-V^{\pi_E}_{R^C}).
\end{equation}
By construction, $\pi_C$ is optimal for $R^C$, so $V^{\hat{\pi}_C}_{R^C}-V^{\pi_C}_{R^C}\leq0$ which implies:
\begin{equation}
V^{\hat{\pi}_C}_{R^C}-V^{\pi_E}_{R^C}\leq(V^{\pi_C}_{R^C}-V^{\pi_E}_{R^E})+(V^{\pi_E}_{R^E}-V^{\pi_E}_{R^C}).
\end{equation}
By construction, we have $V^{\pi_C}_{R^C}=q_{\pi_C}$ and $V^{\pi_E}_{R^E}=q_{\pi_E}$, thus:
\begin{eqnarray}
\rho_E^T(V^{\pi_C}_{R^C}-V^{\pi_E}_{R^E})&=&\rho_E^T(q_{\pi_C}-q_{\pi_E})\\
&=&\sum_{s\in S}\rho_E(s)(q(s,\pi_C(s))-q(s,\pi_E(s)))[\mathds{1}_{\{\pi_C(s)\neq\pi_E(s)\}}].
\end{eqnarray}
Using $\Delta q$, we have:
\begin{equation}
\rho_E^T(V^{\pi_C}_{R^C}-V^{\pi_E}_{R^E})\leq\Delta q\sum_{s\in S}\rho_E(s)[\mathds{1}_{\{\pi_C(s)\neq\pi_E(s)\}}]= \Delta q \epsilon_C.
\end{equation}
Finally, we also have:
\begin{align}
\rho_E^T(V^{\pi_E}_{R^E}-V^{\pi_E}_{R^C})&=\rho_E^T(I-\gamma P_{\pi_E})^{-1}(R^E_{\pi_E}-R^C_{\pi_E}),
\\
&=\rho_E^T(I-\gamma P_{\pi_E})^{-1}\gamma P_{\pi_E}(q_{\pi_C}-q_{\pi_E}),
\\
&=\frac{\gamma}{1-\gamma}\rho_E^T(q_{\pi_C}-q_{\pi_E})\leq \frac{\gamma}{1-\gamma}\Delta q \epsilon_C.
\end{align}
So the upper bound for the second term term is: $\rho_E^T(V^{\hat{\pi}_C}_{R^C}-V^{\pi_E}_{R^C})\leq \frac{\Delta q}{1-\gamma}\epsilon_C$.
If we combine all of the results, we obtain the final bound:
\begin{equation}
0\leq\E_{s\sim\rho_E}[V^*_{\hat{R}^C}(s)-V^{\pi_E}_{\hat{R}^C}(s)]\leq \frac{1}{1-\gamma}(\epsilon_C\Delta q +\epsilon_R(1+C_*))).
\end{equation}
\qed
\end{proof}
A direct corollary of this theorem is that, given a perfect classifier and regressor, CSI produces a reward function for which $\pi_E$ is the \emph{unique} optimal policy.
\begin{corollary}
  \label{cor}
  Assume that $\rho_E>0$ and that the classifier and the regressor are perfect (that is $\epsilon_C = 0$ and $\epsilon_R = 0$). Then $\pi_E$ is the \emph{unique} optimal policy for $\hat R^C$.
\end{corollary}
\begin{proof}

  The function $q$ is the optimal action-value function for $\pi_C$ with respect to the reward $R^C$, by definition (see Eq.~\eqref{eq:rc}).
  As $\epsilon_C = 0$, we have $\pi_C = \pi_E$. This means that $\forall s, \pi_E(s)$ is the only element of the set $\argmax_{a\in A}q(s,a)$. Therefore, $\pi_C = \pi_E$ is the unique optimal policy for $R^C$.
  As $\epsilon_R = 0$, we have $\hat R^C = R^C$, hence the result.
  \qed
\end{proof}
This corollary hints at the fact that we found a non-trivial reward (we recall that the null reward admits \emph{every} policy as optimal). Therefore, obtaining  $\hat{R}_C = 0$ (for which the bound is obviously true: the bounded term is $0$, the bounding term is positive) is unlikely as long as the classifier and the regressor exhibit decent performance.

The only constraints the bound of theorem~\ref{thm} implies on datasets $D_R$ and $D_C$ is that they provide enough information to the supervised algorithms to keep both error terms $\epsilon_C$ and $\epsilon_R$ low. In Sec.~\ref{sec:experiments} we deal with a lack of data in dataset $D_R$. We address the problem with the use of heuristics (Sec.~\ref{subsubsec:heuristics}) in order to show the behavior of the CSI algorithm in somewhat more realistic (but difficult) conditions.

More generally, the error terms  $\epsilon_C$ and $\epsilon_R$ can be reduced by a wise choice for the classification and regression subroutines. The litterature is wide enough for methods accomodating most of use cases (lack of data, fast computation, bias/variance trade-off, etc.) to be found. Being able to leverage such common algorithms as multi-class classifiers and regressors is a big advantage of our cascading approach over existing IRL algorithms.

Other differences between existing IRL or apprenticeship learning approaches and the proposed cascading algorithm are further examined in the following section.

%% Feel good comment
\section{Experiments}
\label{sec:experiments}
In this section, we put our approach under stress to empirically demonstrate its behavior. We begin by providing information pertaining to both benchmarks: an explanation about the amount and source of the available data and a precise description of the way we instanciated the CSI algorithm are given on Sec.~\ref{subsec:instanciation}. 
We then use a classical RL benchmark, the Inverted Pendulum (Sec.~\ref{subsec:IP}), to provide a qualitative illustration of the proposed CSI approach.
We finally provide quantitative results and comparison with state-of-the-art approaches on a highway driving simulator (Sec.~\ref{subsec:highway}).
\subsection{Instanciation}
\label{subsec:instanciation}
\subsubsection{Data scarcity}
\label{subsubsec:data}
In order to create a somewhat realistic environment, we decided that the CSI algorithm could neither use the transition probabilities nor a simulator. The only data it could access is sampled according to the expert policy. This corresponds for example to a situation where a costly system can only be controlled by a trained operator as a bad control sequence could lead to a system breakdown.

More precisely, the expert controls the system for $n$ runs of length $n$, with values of $n$ in $\{3,7,10,15,20\}$, giving $n^2$ samples $(s_k,a_k=\pi_E(a_k),s'_k)_k = D_E$. The dataset $D_C$ fed to the classifier is straightforwardly constructed from $D_E$ by dropping the $s'_k$ terms:
\begin{equation}
  D_C = \{(s_i = s_k,a_i=a_k)_i\}.
\end{equation}
There is no particuliar problem for the classification step. It is however not reasonable to construct the dataset $D_R$ only from expert transitions and expect a small regression error term $\epsilon_R$. Indeed, the dataset $D_E$ only samples the dynamics induced by the expert's policy and not the whole dynamics of the MDP. The regression step also needs samples $((s_j,a_j),\hat r_j)_j$ drawn from non expert actions. We address this problem by using a heuristics.
\subsubsection{Heuristics}
\label{subsubsec:heuristics}
From the raw expert dataset $D_E$ we can construct samples $((s_j=s_k,a_j=a_k),\hat r_j)$ with $\hat r_j$ as specified in Eq.~\eqref{eq:rj}. To compensate for the lack of information about the dynamics of the whole MDP, we assert that disagreeing with the expert's choice is undesirable. Mathematically, we do that by artificially introducting samples $((s_j=s_k,a),r_{min})_{j;\forall a\neq \pi_E(s_j) = a_k}$ where
\begin{equation}
  r_{min} = \min_j\hat r_j - 1.
\end{equation}
Although this heuristics was not analyzed in Sec.~\ref{sec:analysis} (where the availability of a more complete dataset $D_R$ was assumed), the results shown in the next two subsections demonstrate its soundness.
\subsubsection{Subroutines choices}
\label{subsubsec:subroutines}
The cascading algorithm can be instanciated with some standard classification algorithms and any regression algorithm. The choice of such subroutines may be dictated by the kind and amount of available data, by ease of use or by computational complexity, for example.

We referred in Sec.\ref{sec:algo} to \emph{score-function based multi-class classifiers} and explained how the classification rule is similar to the greedy mechanism that exists between an optimal action-value function and an optimal policy in an MDP. Most classification algorithm can be seen as such a classifier. In a simple $k$-nearest neighboor approach, for example, the score function $q(s,a)$ is the number of elements of class $a$ among the $k$-nearest neighboors of $s$. The generic M-SVM model makes the score function explicit (see \cite{guermeur2011generic}). In our experiments, we choose to use a structured margin classification approach \cite{taskar2005learning} that, given a feature vector $\phi: S\times A \rightarrow \mathbb{R}^p$ solves:
\begin{eqnarray*}
  &\min_{\theta,\zeta}\frac{1}{2}\|\theta\|^2 +
  \frac{\eta}{N}\sum_{i=1}^N \zeta_i \\
  &\text{~s.t.~} \forall i,
  \theta^T{\phi}(s_i,a_i)+\zeta_i \geq \max_a \theta^T
  {\phi}(s_i,a) + l(s_i,a). \label{eq:qp_taskar}
\end{eqnarray*}
with $l(s_i,a) = 1\textrm{ if } a\neq a_i \textrm{ and } l(s_i,a_i)=0$. The score function of this classifier is simply $q(s,a) = \theta^T \phi(s,a)$. The implementation of the Structured Classification for IRL (SCIRL) algorithm \cite{klein2012scirl} (see Sec.~\ref{sec:related}) is based on this. We chose to use it in our CSI implementation to provide a fair comparison between both algorithms.

Ease of implementation leads us to choose a simple least-square regressor for our experiments. Using the same features $\phi$ as our classifier, we introduce $\lambda$ a regularization coefficient (set to $0.1$ in our experiments), $I$ the identity matrix and 
\begin{eqnarray}
  \Phi = \begin{bmatrix}
    \vdots\\
    \phi(s_j,a_j)\\
    \vdots
  \end{bmatrix} \in \mathbb{R}^{n|A|\times p}\textrm{ and }
  \hat R = \begin{bmatrix}
    \vdots\\
    \hat r_j\\
    \vdots
  \end{bmatrix}\in \mathbb{R}^{n|A|},
\end{eqnarray}
with $r_j = r_{min}$ if $a_j \neq \pi_E(s_j)$. Our regressor can be written:
\begin{equation}
  \omega = (\Phi^T\Phi + \lambda I)^{-1}\Phi^T\hat R.
\end{equation}
The reward function is: $\hat R^C(s,a) = \omega^T \phi(s,a)$

It is possible to get imaginative in the last step. For example, using a Gaussian process regressor \cite{rasmussen2006gaussian} that outputs both expectation and variance can enable (nonwithstanding a nontrivial amount of work) the use of reward-uncertain reinforcement learning \cite{regan2011robust}. 

Our complete instanciation of CSI is summed up in Alg.~\ref{alg:CSIimp}.
\begin{algorithm}%[H]
    %\small
  \caption{A CSI instanciation with heuristics}
  \label{alg:CSIimp}
  \emph{\textbf{Given}} a dataset  $D_E = (s_k,a_k=\pi_E(a_k),s'_k)_k$\;\\
  \emph{\textbf{Given}} a feature function $\phi: S\times A \rightarrow \mathbb{R}^p$\;\\
  \emph{\textbf{Construct}} the dataset $D_C=\{(s_i=s_k,a_i=\pi^E(s_i))=a_k\}$ \;\\
  \emph{\textbf{Train}} a score function-based classifier on $D_C$, obtaining decision rule $\pi_C$ and score function $q:S\times A \rightarrow \mathbb R$ by solving \begin{eqnarray*}
  &\min_{\theta,\zeta}\frac{1}{2}\|\theta\|^2 +
  \frac{\eta}{N}\sum_{i=1}^N \zeta_i \\
  &\text{~s.t.~} \forall i,
  \theta^T{\phi}(s_i,a_i)+\zeta_i \geq \max_a \theta^T
  {\phi}(s_i,a) + l(s_i,a). \label{eq:qp_taskar}
\end{eqnarray*}\;\\
\emph{\textbf{Construct}} the dataset $\{((s_{j}=s_k,a_{j}=a_k),\hat r_j)_j\}$ with $\hat r_j = q(s_j,a_j) - \gamma q(s'_j=s'_k,\pi_C(s'_j=s'_k))$\;\\
\emph{\textbf{Set}} $r_{min} = \min_j\hat r_j - 1.$\;\\
\emph{\textbf{Construct}} the training set  $D_R = \{((s_{j}=s_k,a_{j}=a_k),\hat r_j)_j\}\cup\{((s_j=s_k,a),r_{min})_{j;\forall a\neq \pi_E(s_j) = a_k}\}$ \;\\
\emph{\textbf{Learn}} a reward function $\hat R^C$ from the trainsing set $D_R$ by computing:
\begin{equation*}
  \omega = (\Phi^T\Phi + \lambda Id)^{-1}\Phi^T\hat R.
\end{equation*}\;\\
\emph{\textbf{Output}} the reward function $\hat R^{C}: (s,a) \mapsto \omega^T \phi(s,a)$ \;
\end{algorithm}

\subsection{Inverted Pendulum}
\label{subsec:IP}
We provide here qualitative results on the inverted pendulum problem. Quantitative results obtained by following a procedure very similar to Sec.~\ref{subsec:highway} were of the same nature and allowed drawing the same conclusions as those provided in Sec.~\ref{subsubsec:hresults}. We thus omit them here for the sake of avoiding redundancy. Instead, we use the inverted pendulum problem for illustrative purposes.
\subsubsection{Setting}
The inverted pendulum (also known as the cart pole) is a classical benchmark in the RL community, it is fully described in \cite{lagoudakis2003least}. The goal is to maintain a pole in a unstable equilibrium position with the help of a cart moving along one axis. The state space consists in the angular position and speed of the pole. It is possible to use an RL algorithm to learn how to balance the pendule in an upright position by using the following reward: as long as the angular position of the pole is (in absolute value) less than $\pi \over 2$ the reward is $0$, the reward is $-1$ if the pole falls; this reward function is plotted on Fig.~\ref{toto}. Three actions are available to the agent: it can choose to apply a leftward or rightward force on the cart or no force at all.

The fact that the state space is continuous is not a problem to our particular instanciation of the CSI approach (see Alg.~\ref{alg:CSIimp}) as long as we provide a feature function $\phi: S\times A \rightarrow \mathbb{R}^p$. Here $p=30$ as we use the exact same feature functions as proposed in \cite{lagoudakis2003least}. The state space is featured with a $3\times 3$ grid of Gaussian functions concatenated with a constant component to construct a feature vector $\psi:S \rightarrow \mathbb{R}^{10}$. The eature function $\phi$ is defined as:
\begin{equation}
  \phi(s,a) = \begin{pmatrix}
     \delta_{a = \textsc{left}}\psi(s)\\
     \delta_{a = \textsc{none}}\psi(s)\\
     \delta_{a = \textsc{right}}\psi(s)
  \end{pmatrix}
\end{equation}


\subsubsection{Results}
The results are shown Fig.~\ref{fig:IP}. In the left column, we show the reward and value functions of the expert. The expert (an RL agent trained on the  reward shown on Fig.~\ref{toto}) is able to balance the pendule indefinitely. With as little as 300 transitions (30 seconds of balancing), CSI is able to find a reward that, once optimized by an RL agent, will reliably lead to a control as good as the expert's. As Fig.~\ref{tata} illustrates, the data we get from the expert only covers a tiny fraction of the state space. Furthermore, the feature function $\phi$ can not permit to accurately approximate the true reward\footnote{The true reward function is a function in $\mathbb{R}^S$, CSI outputs a reward in $\mathbb{R}^{S\times A}$, but we plot here $R(s,\pi^*(s))$ with $\pi^*$ the optimal policy for $R$.}. Despite these two setbacks, CSI manages to find a reward for which not only the expert's control is near-optimal (that is what the analysis of Sec.~\ref{sec:analysis} garantees under the assumption that a reasonable amount of data is available in order to keep both classification and regression errors down) but that also leads to a control policy as good as the expert's \emph{with respect to the (theoretically) unknown expert's reward function}. This phenomenon appears clearly when comparing figures \ref{fifi} and \ref{riri}: although the reward functions were different, the associated optimal value functions are (up to the scale, which does not actually matters) very similar. 

\begin{figure}
  \begin{minipage}{.4\linewidth}
   \subfigure[Reward on which the expert was trained]{\label{toto}\includegraphics[width=1.2\linewidth]{"LAFEM_Exp3_true_R"}}
   \end{minipage}
  \hfill
  \begin{minipage}{.4\linewidth}
    \subfigure[Reward found by CSI. The green zone shows where the data available to the algorithm lies in the state space.]{\label{tata}\hspace{-3em}\includegraphics[width=1.2\linewidth]{"LAFEM_Exp3_lafem_R"}}
  \end{minipage}\\
  \begin{minipage}{.4\linewidth}
    \subfigure[Value function of the expert (i.e. optimal value function for the reward displayed Fig.~\ref{toto})]{\label{riri}\includegraphics[width=1.2\linewidth]{"LAFEM_Exp3_Vexpert"}}
  \end{minipage}
  \hfill
  \begin{minipage}{.4\linewidth}
  \subfigure[Optimal value function for the reward function found by CSI (Fig.~\ref{tata})]{\label{fifi}\includegraphics[width=1.2\linewidth]{"LAFEM_Exp3_Vagent"}}
  \end{minipage}
  \caption{Results on the inverted pendulum benchmark, illsutrating the dissimilarity of the rewards, but the similarity of the value functions. The state space is represented on a position-speed plan.}
  \label{fig:IP}
\end{figure}


\subsection{Highway driving simulator}
\label{subsec:highway}
Let's now move from a qualitative illustration to a quantitative assessment of the performance of the CSI algorithm. The Highway driving problem is described Sec.~\ref{subsubsec:hsetting}, the criterion we use is explained Sec.~\ref{subsubsec:hcriterion} and finally results and comparison to other approaches are presented Sec.~\ref{subsubsec:hresults}.
\subsubsection{Setting}
\label{subsubsec:hsetting}
The setting of the experiment is a driving simulator inspired from a benchmark already used in \cite{syed2008apprenticeship,syed2008game}. The agent controls a car that can switch between the three lanes of the road, go off-road on either side and modulate between three speed levels. At all timesteps, there will be one car in one of the three lanes. Even at the lowest speed, the player's car moves faster than the others. When the other car disappears at the bottom of the screen, another one appears at the top in a randomly chosen lane. It takes two transitions to completely change lanes, as the player can move left or right for half a lane's length at a time. At the highest speed setting, if the other car appears in the lane the player is in, it is not possible to avoid the collision. The main difference between the original benchmark \cite{syed2008apprenticeship,syed2008game} and ours is that we made the MDP more ergodic by allowing the player to change speed whenever he wishes so, not just during the first transition. If anything, by adding two actions, we enlarged the state-action space and thus made the problem tougher.
%FIXME:METTRE UNE CAPTURE DECRAN.
The reward function $R_E$ trains the expert to go as fast as possible (high reward) while avoiding collisions (harshly penalised) and avoiding going off-road (moderately penalised). Any other situation receives a null reward.

The feature function $\phi$ for this setting is the natural tabular feature function where each state action couple $(s,a)$ is associated with a unique index $i_{(s,a)}$. The only non-zero component of $\phi(s,a)$ is then the component at index $i_{(s,a)}$.

We need some baseline to compare our new approach to. The natural random baseline consists in drawing a random reward vector (with a uniform law) and training an agent on it. Another natural benchmark we use is to compute the mean value function of the classifier policy. We also compare the proposed cascading method to the projection version of the seminal algorithms in \cite{abbeel2004apprenticeship}\footnote{We call it PIRL for Projection IRL.} and the SCIRL algorithm of \cite{klein2012scirl}. The random baseline and PIRL do not use samples (with obvious reasons for the random baseline). PIRL is instanciated in a way that makes use of the transition probability matrix $P$ to exactly compute all relevant objects (feature expectations, optimal policies) via exact dynamic programming. It is left to run for 70 iterations. We could have displayed PIRL's data sensitivity by making $\pi_E$ known only through observations, but we decided against it because we want this baseline to show the performance of a fully informed well-established apprenticeship algorithm.

Apart from SCIRL, to the best of our knowledge, no other algorithm can infer a reward function only from expert data. We will see in Sec.~\ref{sec:related} that most existing approaches either need to solve the direct RL problem repeatedly (which can not be done on this setting from expert data alone) or rely on a sampling of the dynamics of the whole MDP and do not propose (as CSI or SCIRL) the use of heuristics to avoid this problem. The reward functions found by SCIRL and CSI are then optimized using a dynamic programming algorithm to show the performance of the associated policy.

\subsubsection{Criterion}
\label{subsubsec:hcriterion}
The theoretical bound provided in Sec.~\ref{sec:analysis} is about the optimality of the expert with respect to the reward function computed by our algorithm (which is also the way the problem was framed in Sec.~\ref{sec:background}). For this experiment, we use a somewhat more restrictive criterion: a measure of the optimality of an agent trained on the reward output by our algorithm, with respect to the ``true'' reward function, the reward function of the expert. We use the mean (over the uniform distribution) value function with respect to $R_E$ of the policy $\hat \pi_C$ learned by optimizing $\hat R^C$: $\mathbf{E}_{s\sim\mathcal{U}}[V^{\pi}_{R_E}(s)]$. We do that firstly because, for once, we can: the bound we provide in the analysis is something that can be measured even if $R_E$ does not actually exist. We make use here of the fact that $R_E$ not only exists, but is known to us as we use, as an expert, a RL agent perfectly trained on a handcrafted $R_E$. The second reason for this different metric is that we believe the theoretical bound to be convincing, and what we wanted to illustrate here is not that the theoretical bound holds, but that under really dire sample scarcity (a case not covered by the analysis) our algorithm is well-behaved given the use of heuristics. Finally, it is quite fair to use a single objective criterion than to just compare all approaches with one another.

\subsubsection{Results}
\label{subsubsec:hresults}
Results are shown Fig.~\ref{fig:Highway}. We give the values of $\mathbf{E}_{s\sim\mathcal{U}}[V^{\pi}_{R_E}(s)]$ with $\pi$ being in turn the optimal policy for the rewards given by SCIRL and CSI, the policy $\pi_C$ of the classifier (the very one the classification step of CSI outputs), the output of the PIRL algorithm, the expert's policy $\pi_E$ and the optimal policy for a randomly drawn reward. Performance for CSI is on par with the PIRL and SCIRL algorithms, slightly below the performance of the expert himself. Very few samples (100) are needed to reliably achieve expert-level performance.

It is very interesting to compare our algorithm to the behavior of a classifier alone (respectively red and green plots on Fig.~\ref{tutu}). With \emph{the exact same data}, albeit the use of a very simple heuristics, the cascading approach demonstrates far better performance from the start. This is a clear illustration of the fact that using the Bellman equation to construct the data fed to the regressor and outputting not a policy, but a reward function that can be optimized on the MDP truly makes use of the information that the transitions $(s,a,s')$ bear (we recall that the classifier only uses $(s,a)$ couples).

Furthermore, the classifier whose results are diplayed here is the output of the first step of the algorithm. The classification performance is obviously not that good, this hints to the fact that our algorithm may be empirically more forgiving of classification errors than our theoretical bound let us expect.

PIRL is given the transition probabilities (that is, a $|S|\times |S|\times |A|$ matrix) and yet it does only slightly better than our approach with 100 samples. This also shows that CSI is indeed very sample efficient, without having to repeatedly solve the MDP. Furthermore, it is much faster to run our implementation of CSI on so little data than to run PIRL with the dynamics for 70 iterations.

Finally, only the SCIRL algorithm operates on the same data as our proposed approach, with very slightly better results.

\begin{figure}
   \subfigure[PIRL and Random Baselines. There is no abcissa as these methods do not need data input. PIRL uses the transition probabilities matrix $P$.]{\includegraphics[width=.5\linewidth]{"Fig1-NoX"}}
   \subfigure[Cascading and Classification.]{\label{tutu}\includegraphics[width=.5\linewidth]{"Fig2-ClassifAndUs"}}
  \subfigure[SCIRL and Cascading mean value.]{\includegraphics[width=.5\linewidth]{"Fig3-SCIRLAndUs"}}
  \subfigure[Legend]{\includegraphics[width=.5\linewidth]{"Fig4-Legend"}}

  \caption{Data is shown with mean, standard deviation, minimum and maximum value over 50 runs. See Sec.~\ref{subsubsec:hsetting} for detailled information about the experimental setting.}
  \label{fig:Highway}
\end{figure}


\section{Related Work}
\label{sec:related}
IRL was first introduced in \cite{russell1998learning} and then formalized in \cite{ng2000algorithms}. Approaches summarized in \cite{neu2009training} can be seen as iteratively contructing a reward function, solving an MDP at each iteration. Some of these algorithms are IRL algorithms while others fall in the AL category, as for example the projection version of the algorithm in \cite{abbeel2004apprenticeship} which we used in our benchmarks (Sec.~\ref{sec:experiments}). In both cases the need to solve an MDP at each step may be very demanding, both sample-wise and computationally. CSI being able to output a reward function without having to solve the MDP is thus a significant improvement.

AL via classification has been proposed in for example \cite{ratliff2007imitation} with the help of a structured margin method. Using the non trivial notion of metric in an MDP, the authors of \cite{melo2010learning} build a kernel which is used
in a classification algorithm, showing improvements compared to a
non-structured kernel.

Classification and IRL have met in the past in \cite{ratliff2006maximum}, but the labels were complete optimal policies rather than actions and the inputs were MDPs, which had to be solved. The SCIRL (for Structured Classification for IRL) algorithm \cite{klein2012scirl} which we used Sec.~\ref{subsec:highway} is able to solve the IRL problem (without having to solve any MDP) in the same dire conditions the cascading approach has been tested in. It injects the structure of the MDP in a linearly parametrized score-function based classifier by using the expert's feature expectation (the expected discounted sum of the a feature function evaluated over the states of the trajoctories the expert follow) as feature function. This necessity of a linarly parametrized approach and the need to compute the expert's feature expectations makes the SCIRL algorithm less convenient than the CSI approach.

Few IRL or AL algorithms do not require solving an MDP. The approach of \cite{syed2008apprenticeship} requires knowing the transition probabilities of the MDP (which CSI does not need) and outputs a policy (and not a reward). The algorithm in \cite{dvijotham2010inverse} only applies to linearly-solvable MDPs whereas our approach does not place such restrictions. Closer to our use-case is the idea presented in \cite{boularias2011relative} to use a subgradient ascent of a utility function based on the notion of relative entropy. Importance sampling is suggested as a way to avoid solving the MDP. This requires sampling trajectories according to a non-expert policy and
the direct problem remains at the core of the approach (even if
solving it is avoided).


\section{Conclusion}
\label{sec:conclusion}
We have introduced a new way to perform IRL by cascading two supervised approaches. The expert is theoretically shown to be near-optimal for the reward function the proposed algorithm outputs, given small classification and regression errors. Practical examples of classifiers and regressors have been given, and one combination has been empirically (on two classic benchmarks) shown to be very resilient to dire lack of data on the input (only data from the expert was used to retrieve the reward function), provided the use of simple heuristics. On one of the benchmarks (a car driving simulator), our algorithm is shown to outperform other state of the art approaches, and to be on-par with the very recent SCIRL algorithm. We plan on deepening the analysis of the theoretical properties of our approach and on applying it to real world robotics problems.

\bibliographystyle{plain}
\bibliography{Biblio}

\end{document}
